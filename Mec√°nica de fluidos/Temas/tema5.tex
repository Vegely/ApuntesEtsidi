\chapter{Flujos de viscosidad dominante.}
\section{Introducción.}
Se trabaja con líquidos newtonianos donde se cumple que:
\[\rho, \mu =cte\]
\[\vec{\nabla}\cdot\vec{v}=0\]
\[\rho\dfrac{\partial \vec{v}}{\partial t}
+
\red{\underbrace{\black \rho\left(\vec{v}\cdot\vec{\nabla}\right)\vec{v}}_{\text{Inercia convectiva}}} \black
=
-
\vec{\nabla}P
+
\red{\underbrace{\black \mu\vec{\nabla}^2\vec{v} }_{\text{Esfuerzos viscosos}}} \black
+
 \vec{f}_v\]


Para conocer si el flujo es de viscosidad dominante se emplea el número de Reynolds, que como se definió en temas anteriores:
\[Re=\dfrac{\text{Orden de magnitud de la inercia convectiva}}{\text{Orden de magnitud de fuerzas viscosas}}=\dfrac{\rho v_c L_c}{\mu}\]
\begin{itemize}
	\item Si $Re\uparrow\uparrow$ efectos viscosos despreciables.
	\item Si $Re\downarrow\downarrow$ efectos viscosos dominantes.
\end{itemize}

\section{Fluidos unidireccionales.}
\section{Flujo de Covette.}
\section{Flujo de Poiseulle.}
\section{Flujo de Hagen-Poiseuille.}
\section{Espesor de capa límite.}