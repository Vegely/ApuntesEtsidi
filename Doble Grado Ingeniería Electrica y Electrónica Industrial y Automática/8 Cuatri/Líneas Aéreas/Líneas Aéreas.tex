\documentclass[12pt,a4paper]{article}
% Set the font (output) encodings
\usepackage[T1]{fontenc}% Permite el uso de fuentes modernas
\usepackage{fontspec}
% Spanish-specific commands
\usepackage[spanish]{babel}
\usepackage[a4paper, margin=3cm]{geometry} % Cambiar márgenes
\setlength{\parindent}{0pt} % elimna sangría para todo el documento
\usepackage{listings} % Paquete para incluir código fuente
\usepackage{amsmath}  % Para formato matemático, si es necesario
\usepackage{amssymb}  % Para \mathbb{N}
\usepackage{booktabs} % Para líneas de tabla más profesionales
\usepackage{enumitem} % Control de espacio en listas
\usepackage[hidelinks]{hyperref} %referencias del índice invisibles
\usepackage{xcolor}   % Paquete para manejar colores
\usepackage{graphicx} %LaTeX package to import graphics
\graphicspath{{imágenes/}}% configuring the graphicx package
\usepackage{float} %imagenes
\usepackage{array}          % Mejora el control de tablas
\usepackage{tabularx} % Asegúrate de incluir este paquete en el preámbulo de tu documento LaTeX
\usepackage{makecell}
\usepackage[table]{xcolor}  % Soporte para colores en tablas
\usepackage{changepage} % Permite modificar márgenes localmente
\usepackage{circuitikz}
\usepackage{multirow}
\usepackage{fancybox}
\usepackage{mdframed}

\begin{document}	
	\begin{titlepage} % portada
		\centering
		\vspace*{2cm} % Espaciado vertical antes del título
		{\Huge \textbf{Líneas Aéreas de Alta Tensión}}\\[1cm] % Título del documento en letra grande y negrita
		\vfill % Espaciado flexible
		\vspace*{2cm} % Espaciado vertical después del título
	\end{titlepage}

\section{Formulario.}
\subsection{Parámetros eléctricos de la línea.}
\subsubsection{Resistencia.}	
A la temperatura de 20 °C:
\[
\boxed{R_{cc20°C} = \frac{\rho_{20°C}}{S} \quad (\Omega/km)}
\]
- \( S \), sección (mm²)

- \( \rho_{Cu} \) resistividad del cobre, \( 17.241 \, (\Omega \cdot mm^2/km) \)

- \( \rho_{Al} \) resistividad del aluminio, \( 28.264 \, (\Omega \cdot mm^2/km) \)
\vspace{0.2cm}

Para una temperatura \(\theta\) cualquiera:
\[
\boxed{R_{cc\theta} = R_{cc20°C} \cdot [1 + \alpha \cdot (\theta - 20)]}
\]
\[
\alpha_{20°Cu} = 0.00393 \, ^\circ C^{-1} \qquad \alpha_{20°Al} = 0.00403 \, ^\circ C^{-1}
\]
Resistencia en CA:
\[
\boxed{R_{ca} \approx R_{cc\theta} (1 + Y_s) = R_{ccc} \left( 1 + 7.5 \cdot f^2 \cdot d^4 \cdot 10^{-7} \right) \quad (\Omega / km)}
\]
\subsubsection{Reactancia inductiva serie.}
\[
\boxed{X = \omega \cdot L_k \cdot l \quad (2)}
\]
Expresión general para la inductancia \textbf{por fase} de una línea se expresa:
\[
\boxed{L_k = 2 \cdot 10^{-4} \ln \frac{DMG}{RMG'} \quad (H / km)}
\]
Siendo:\\
- \( DMG \), la distancia media geométrica.\\
- \( RMG' \), el radio medio geométrico.
\[ 
\boxed{RMG' = \sqrt[n]{e^{-1/4} \cdot r \cdot n \cdot \left( \frac{\Delta}{2 \cdot \sin(\pi / n)} \right)^{n-1}}}
\]
\begin{enumerate}
	\item Líneas de simple circuito simplex (un conductor por fase).
	\[
	\boxed{	L_k = 2 \cdot 10^{-4} \cdot \ln \frac{DMG}{r'} \quad (H/km)}
	\]
	\[
	\boxed{DMG = \sqrt[3]{D_{12} \cdot D_{23} \cdot D_{31}}}
	\]
	\[
	\boxed{r' = e^{-1/4} \cdot r \quad \text{siendo \( r \), el radio del conductor}}
	\]
	\[
	\boxed{X = \omega \cdot L_k \cdot l \quad (\Omega)}
	\]
	\item Líneas de simple circuito dúplex (2 conductores por fase).
	\[
	\boxed{	L_k = 2 \cdot 10^{-4} \, \text{ln} \frac{DMG}{RMG'} \quad (H / km)}
	\]
	\[
	\boxed{	DMG = \sqrt[3]{D_{12} \cdot D_{23} \cdot D_{31}}}
	\]
	\[
	\boxed{	RMG = \sqrt{r' \cdot \Delta}}
	\]
	\[
	\boxed{r' = e^{-1/4} \cdot r \quad \text{siendo \( r \), el radio del conductor}}
	\]
	\[
	\boxed{	X = \omega \cdot L_k \cdot l \quad (\Omega)}
	\]
	\item Líneas de doble circuito simplex (un conductor por fase).
	\[
	\boxed{	L_k = 2 \cdot 10^{-4} \cdot \ln \left(\frac{DMG_{ff}}{DMG_f'}\right) \quad (H/km)}
	\]
	La inductancia por circuito sería: $\boxed{2 \cdot L_k}$ 
	\[
	\boxed{DMG_{ff} = \sqrt[3]{(D_{12}\cdot D_{12'}\cdot D_{1'2}\cdot D_{1'2'})^\frac{1}{4} \cdot (D_{23}\cdot D_{23'}\cdot D_{2'3}\cdot D_{2'3'})^\frac{1}{4} \cdot (D_{31}\cdot D_{31'}\cdot D_{3'1}\cdot D_{3'1'})^\frac{1}{4}}} 
	\]
	\[
	\boxed{DMG_f = \sqrt[6]{\left( r' \cdot D_{11'} \right) \cdot \left( r' \cdot D_{22'} \right) \cdot \left( r' \cdot D_{33'} \right)}}
	\]
	\[
	\boxed{	r' = e^{-1/4} \cdot r \quad \text{siendo } r, \text{ el radio del conductor.}}
	\]
	\[
	\boxed{	X = \omega \cdot L_k \cdot l \quad (\Omega)}
	\]
	\item Líneas de doble circuito duplex (dos conductores por fase).
	\[
	\boxed{	L_k = 2 \cdot 10^{-4} \cdot \ln \left(\frac{DMG_{ff}}{DMG_f'}\right) \quad (H/km)}
	\]
	La inductancia por circuito sería: $\boxed{2 \cdot L_k}$ 	
	\[
	\boxed{DMG_{ff} = \sqrt[3]{(D_{12}\cdot D_{12'}\cdot D_{1'2}\cdot D_{1'2'})^\frac{1}{4} \cdot (D_{23}\cdot D_{23'}\cdot D_{2'3'})^\frac{1}{4} \cdot (D_{31}\cdot D_{31'}\cdot D_{3'1}\cdot D_{3'1'})^\frac{1}{4}}} 
	\]
	\[
	\boxed{	DMG_f' = \left(RMG'\right)^{1/2} \cdot \left(D_{11'} \cdot D_{22'} \cdot D_{33'}\right)^{1/6}}
	\]
	\[
	\boxed{	RMG' = \sqrt{r'\cdot \Delta}  \quad \text{con} \quad r' = e^{-1/4} \cdot r \quad \text{siendo } r, \text{ el radio del conductor.}}
	\]
	\[
	\boxed{	X = \omega \cdot L_k \cdot l \quad (\Omega)}
	\]
\end{enumerate}
\subsubsection{Susceptancia capacitiva paralelo (S/km).}
\[
\boxed{	 B = \frac{1}{X_c} \cdot 1 = \frac{1}{\frac{1}{\omega \cdot C_k}} \cdot 1 = \omega \cdot C_k \cdot 1 \quad (S)}
\]
Expresión general para la inductancia \textbf{por fase} de una línea se expresa:
\[
\boxed{	C_k = \frac{0.0556 \cdot 10^{-6}}{\ln \left( \frac{DMG}{RMG} \right)} \quad \left( \frac{F}{km} \right)}
\]
Siendo: \\
- DMG, la distancia media geométrica entre ejes de fases (m). \\
- RMG', el radio medio geométrico del conductor (m).
\[
\boxed{	RMG = \sqrt[n]{r \cdot n \left( \frac{\Delta}{2 \cdot \sin(\pi / n)} \right)^{n-1}}}
\]
\begin{enumerate}
	\item Líneas de simple circuito simplex (un conductor por fase).
	\[
	\boxed{	C_k = \frac{0,0556 \cdot 10^{-6}}{\ln \frac{DMG}{r}} \left( \frac{F}{\text{km}} \right)}
	\]
	\[
	\boxed{	DMG = \sqrt[3]{D_{12} \cdot D_{23} \cdot D_{31}} \quad \text{siendo \( r \), el radio del conductor}}
	\]
	\[
	\boxed{	B = \omega \cdot C_k \cdot l \quad (S)}
	\]
	\item Líneas de simple circuito duplex (dos conductores por fase).
	\[
	\boxed{	C_k = \frac{0,0556 \cdot 10^{-6}}{\ln \frac{DMG}{RMG}} \left( \frac{F}{km} \right)}
	\]
	\[
	\boxed{	DMG = \sqrt[3]{D_{12} \cdot D_{23} \cdot D_{31}}}
	\]
	\[
	\boxed{	RMG = \sqrt{r \cdot \Delta} \quad \text{siendo \( r \), el radio del conductor}}
	\]
	\[
	\boxed{	B = \omega \cdot C_k \cdot l \quad (S)}
	\]
	\item Líneas de doble circuito simplex (un conductor por fase).
	\[
	\boxed{	C_k = \frac{0,0556 \cdot 10^{-6}}{\ln \frac{DMG_{ff}}{DMG_f}} \left( \frac{F}{km} \right)}
	\]
	La capacidad de un solo circuito sería: $\boxed{\frac{C_k}{2}}$
	\[
	\boxed{DMG_{ff} = \sqrt[3]{(D_{12}\cdot D_{12'}\cdot D_{1'2}\cdot D_{1'2'})^\frac{1}{4} \cdot (D_{23}\cdot D_{23'}\cdot D_{2'3'})^\frac{1}{4} \cdot (D_{31}\cdot D_{31'}\cdot D_{3'1}\cdot D_{3'1'})^\frac{1}{4}}} 
	\]
	\[
	\boxed{DMG_f = \left[ \left( r \cdot D_{11'} \right) \left( r \cdot D_{22'} \right) \cdot \left( r \cdot D_{33'} \right) \right]^{1/6} \quad \text{siendo \( r \), el radio del conductor}}
	\]
	\[
	\boxed{B = \omega \cdot C_k \cdot l \quad (S)}
	\]
	\item Líneas de doble circuito duplex (dos conductores por fase).
	\[
	\boxed{	C_k = \frac{0,0556 \cdot 10^{-6}}{\ln \frac{DMG_{ff}}{DMG_f}} \left( \frac{F}{km} \right)}
	\]
	La capacidad de un solo circuito sería: $\boxed{\frac{C_k}{2}}$
	\[
	\boxed{DMG_{ff} = \sqrt[3]{(D_{12}\cdot D_{12'}\cdot D_{1'2}\cdot D_{1'2'})^\frac{1}{4} \cdot (D_{23}\cdot D_{23'}\cdot D_{2'3'})^\frac{1}{4} \cdot (D_{31}\cdot D_{31'}\cdot D_{3'1}\cdot D_{3'1'})^\frac{1}{4}}} 
	\]
	\[
	\boxed{	DMG_f = (RMG)^{1/2} \cdot (D_{11'} \cdot D_{22'} \cdot D_{33'})^{1/6}}
	\]
	\[
	\boxed{	RMG = \sqrt{r \cdot \Delta} \quad \text{siendo } r, \text{ el radio del conductor.}}
	\]
	\[
	\boxed{	B = \omega \cdot C_k \cdot 1 \quad (S)}
	\]
\end{enumerate}
\subsubsection{Conductancia (S/km).}	
Se define la conductancia por unidad de longitud como la inversa de la resistencia de aislamiento.
\[
\boxed{G_k = \frac{1}{R_{\text{ais}}} \left( \frac{S}{\text{km}} \right)}
\]
Para líneas de tensión superior a 132 kV, el valor de la conductancia es del orden:

- Tiempo seco:  
\[
\boxed{G = 1 \times 10^8 - 10 \times 10^8 \, \text{S/km}}
\]
- Tiempo húmedo:  
\[
\boxed{G' = 10 \times 10^8 - 30 \times 10^8 \, \text{S/km}}
\]
La pérdida de potencia, por fase, originada por el paso de la corriente de fuga, a través de la resistencia del aislamiento es:
\[
\boxed{P_{\text{ais}} = R_{\text{ais}} \cdot I_{\text{fuga}}^2 = R_{\text{ais}} \cdot \left( \frac{U/\sqrt{3}}{R_{\text{ais}}} \right)^2 = \frac{U^2}{3} \cdot G_k \cdot 10^6 \left( \frac{W}{\text{km}} \right) \quad \text{siendo } U \text{ kV, y } G_k \, (\text{S/km})}
\]


	
\end{document}	