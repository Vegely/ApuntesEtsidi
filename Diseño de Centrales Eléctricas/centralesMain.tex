\documentclass{book}

\usepackage[spanish]{babel}
\usepackage{amsmath}
\usepackage{graphicx}
\usepackage{subfigure}
\usepackage{listings}
\usepackage{tikz}
\usepackage{circuitikz}
\usetikzlibrary{babel}
\usepackage{array}
\usepackage{makecell}
\usepackage{tabularray}
\usepackage{subcaption}
\usepackage{booktabs}
\usepackage{siunitx}
\usepackage{pgfplots}
\usepackage{pgfplotstable}
\usepgfplotslibrary{colormaps}
\pgfplotsset{compat=1.17}
\usepgfplotslibrary{groupplots}

\usepackage{multicol}
\usepackage{multirow}
\usepackage{float}
\usepackage{xcolor}
\usepackage{pgf-pie}
\usepackage{pgffor}
\usetikzlibrary{arrows}

\usepackage[T1]{fontenc}
\usepackage{tgpagella}

\usepackage[a4paper, top=2cm, bottom=2cm, left=3cm, right=3cm, marginparwidth = 1.75cm]{geometry}

\graphicspath{ {./res/tema1/} }

\setlength\parindent{12pt}

\title{Apuntes de Diseño de Centrales Eléctricas}
\author{Bogurad Barañski Barañska \and Adrián Teixeira de Uña}
\date{\today}

\def\innerradius{3cm}
\def\outerradius{4cm}

% Macro principal para el gráfico de donut
\newcommand{\donutchart}[1]{
	% Calcula el total
	\pgfmathsetmacro{\totalnum}{0}
	\foreach \value/\colour/\name in {#1} {
		\pgfmathparse{\value + \totalnum}
		\global\let\totalnum=\pgfmathresult
	}
	
	\begin{tikzpicture}
		% Calcula el grosor y la línea media del gráfico
		\pgfmathsetmacro{\wheelwidth}{\outerradius - \innerradius}
		\pgfmathsetmacro{\midradius}{(\outerradius + \innerradius) / 2}
		
		% Gira el gráfico para comenzar desde la parte superior
		\begin{scope}[rotate=90]
			% Bucle a través de cada conjunto de valores
			\pgfmathsetmacro{\cumnum}{0}
			\foreach \value/\colour/\name in {#1} {
				\pgfmathsetmacro{\newcumnum}{\cumnum + \value / \totalnum * 360}
				% Calcula el valor porcentual
				\pgfmathsetmacro{\percentage}{\value / \totalnum * 100}
				% Calcula el ángulo medio de los segmentos de color para colocar las etiquetas
				\pgfmathsetmacro{\midangle}{-(\cumnum + \newcumnum) / 2}
				% Esto es necesario para que las etiquetas se alineen correctamente
				\pgfmathparse{(-\midangle < 180 ? "west" : "east")}
				\edef\textanchor{\pgfmathresult}
				\pgfmathsetmacro\labelshiftdir{1 - 1 * (-\midangle > 180)}
				% Dibuja los segmentos de color
				\fill[\colour] (-\cumnum:\outerradius) arc (-\cumnum:-(\newcumnum):\outerradius) -- (-\newcumnum:\innerradius) arc (-\newcumnum:-(\cumnum):\innerradius) -- cycle;
				% Establece el ángulo acumulado antiguo al nuevo valor
				\global\let\cumnum=\newcumnum
			}
		\end{scope}
	\end{tikzpicture}
}

\begin{document}
	
	\mainmatter	
		\maketitle
		
\section{Tema 1: Introducción y generalidades sobre metrología}
\subsection{Definiciones}
\begin{enumerate}
	\item \underline{\textbf{Magnitud}}: 
	\item \underline{\textbf{Magnitud básica}}:
	\item \underline{\textbf{Magnitud derivada}}:
	\item \underline{\textbf{Unidad de medida}}:
	\item \underline{\textbf{Unidad coherente}}:
	\item \underline{\textbf{Sistema de unidades}}:
	\item \underline{\textbf{Valor de una magnitud}}:
	\item \underline{\textbf{Valor verdadero}}:
	\item \underline{\textbf{Valor convencionalmente verdadero}}:
	\item \underline{\textbf{Medida}}:
	\item \underline{\textbf{Medición general}}:
	\item \underline{\textbf{Medición metrológica}}:
	\item \underline{\textbf{Mensurando}}:
	\item \underline{\textbf{Magnitud de influencia}}:
	\item \underline{\textbf{Señal de medida}}:
	\item \underline{\textbf{Instrumento de medida}}:
	\item \underline{\textbf{Cadena de medida}}:
	\item \underline{\textbf{Valor nominal}}:
	\item \underline{\textbf{Campo de medida}}:
	\item \underline{\textbf{Rango de medida}}:
	\item \underline{\textbf{Constante de medida}}:
	\item \underline{\textbf{Estabilidad}}:
	\item \underline{\textbf{Transparencia}}:
	\item \underline{\textbf{Deriva}}:
	\item \underline{\textbf{Zona muerta}}:
	\item \underline{\textbf{Sensibilidad}}:
	\item \underline{\textbf{Resolución}}:
	\item \underline{\textbf{Veracidad}}:
	\item \underline{\textbf{Precisión}}:
	\item \underline{\textbf{Exactitud}}:
	\item \underline{\textbf{Sesgo}}:
	\item \underline{\textbf{Linealidad}}:
	\item \underline{\textbf{Índice de clase}}:
	\item \underline{\textbf{Incertidumbre de medida}}:
	\item \underline{\textbf{Error de medida }}:
	\item \underline{\textbf{Error aleatorio}}:
	\item \underline{\textbf{Error sistemático}}:
	\item \underline{\textbf{Tolerancia}}:
	\item \underline{\textbf{Incertidumbre}}:
\end{enumerate}
\subsection{Causas de errores}
\subsection{Ley propagación de incertidumbres}
\subsection{Estimación de la incertidumbre}
\subsection{Relación tolerancia incertidumbre}
\subsection{Relación incertidumbre resolución}
	
\end{document}