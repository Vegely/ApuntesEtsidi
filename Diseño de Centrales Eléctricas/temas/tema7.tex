\chapter{Sistemas de excitación. Control tensión - reactiva.}
	\section{Respuesta de sistemas.}
		\subsection{Sistemas de primer orden.}
			Sea el sistema con realimentación unitaria de la figura siguiente:
			\begin{figure}[H]
				\centering
					\begin{circuitikz}[scale=0.7]
						\tikzstyle{every node}=[font=\normalsize]
						\draw  (9.75,14) circle (0.5cm);
						\draw [->, >=Stealth] (7.75,14) -- (9.25,14)node[pos=0.25,above]{X(s)};
						\draw [->, >=Stealth] (10.25,14) -- (11.25,14);
						\draw  (11.25,14.75) rectangle  node {\normalsize G(s)} (13.5,13.25);
						\draw [short] (14.5,14) -- (14.5,12.5);
						\draw [short] (14.5,12.5) -- (9.75,12.5);
						\draw [->, >=Stealth] (9.75,12.5) -- (9.75,13.5);
						\node [font=\normalsize] at (9,14.25) {+};
						\node [font=\normalsize] at (10,13.25) {-};
						\node at (14.5,14) [circ] {};
						\draw [->, >=Stealth] (13.5,14) -- (15.5,14)node[pos=0.5,above]{Y(s)};
					\end{circuitikz}
			\end{figure}
			
			$G(s)$ se denomina "función de transferencia de la planta", y su comportamiento viene descrito por los polos y ceros que lo componen:
			\[G(s) = \dfrac{\prod ceros}{\prod polos} = \dfrac{b_0 + b_1 s + b_2 s^2\dots + b_m s^m}{a_0 + a_1 s + a_2 s^2\dots + a_n s^n}\]
			El sistema será estable si, en la función de transferencia global, el número de polos es mayor al de ceros: $n>m$.
			
			La función de transferencia del sistema es:
			\[M(s) = \dfrac{Y(s)}{X(s)} = \dfrac{G(s)}{1 + G(s)}\]
			
			
			Si una función $G(s)$ es de primer orden es de la forma:
			\[G(s) = \dfrac{K}{1+Ts} = \dfrac{\dfrac{K}{T}}{s+\dfrac{1}{T}}\]
			Donde $K$ es la ganancia del sistema (su valor cuando finaliza el transitorio) y $T$ la constante de tiempo del sistema.
			
			
			La respuesta a escalón de un sistema de primer orden es:
			\[y(t) = K\left(1-e^{-t/T}\right)\]
			
			
			Los sistemas de primer orden tienen un polo. Si el polo es positivo entonces el sistema es inestable, si es negativo es estable y si es 0 el sistema no converge y es críticamente estable:
			\begin{figure}[H]
				\begin{minipage}{0.5\textwidth}
					\begin{figure}[H]
						\centering
						\begin{circuitikz}
							\tikzstyle{every node}=[font=\normalsize]
							\draw [->, >=Stealth] (11.25,13) -- (15.5,13)node[pos=1,above]{Re};
							\draw [->, >=Stealth] (13.25,12.5) -- (13.25,15)node[pos=1,above]{Im};
							\node at (12.25,13) {$\times$};
							\draw [ color={rgb,255:red,0; green,128; blue,0}, ->, >=Stealth, dashed] (13,14.5) -- (11.25,14.5)node[pos=0.5,above]{Estable};
							\draw [ color={rgb,255:red,255; green,0; blue,0}, ->, >=Stealth, dashed] (13.5,14.5) -- (15.5,14.5)node[pos=0.5,above]{Inestable};
						\end{circuitikz}
						
						\label{fig:my_label}
					\end{figure}
				\end{minipage}
				\begin{minipage}{0.5\textwidth}
					\begin{figure}[H]
						\centering
						\begin{circuitikz}[scale = 0.8]
							\tikzstyle{every node}=[font=\normalsize]
							\draw [->, >=Stealth] (13.75,12.75) -- (17.5,12.75)node[pos=1,above]{t};
							\draw [->, >=Stealth] (14,12.5) -- (14,16)node[pos=1,right]{y};
							\draw [ color={rgb,255:red,0; green,128; blue,255}, short] (14,12.75) .. controls (14.25,14) and (14.25,15.5) .. (17.25,15.5);
							\draw [dashed] (17.25,15.5) -- (14,15.5)node[pos=1,left]{K};
						\end{circuitikz}
						
						\label{fig:my_label}
					\end{figure}
				\end{minipage}
			\end{figure}
			
			
			
			
		\subsection{Sistemas de segundo orden subamortiguados.}
			Sea el sistema con realimentación unitaria de la figura anterior. Si ahora $G(s)$ tiene 2 polos será de segundo orden. La posición de los polos de la función de transferencia del lazo abierto determina el lugar de las raíces, que muestra el comportamiento del sistema: si son complejos conjugados será subamortiguado.
			
			\begin{figure}[H]
				\begin{minipage}{0.5\textwidth}
					\[G(s) = \dfrac{K \omega_n^2}{s^2 + 2\xi \omega_n s + \omega_n^2}\]
					\[\sigma = \xi \omega_n \qquad \xi = \cos \theta\]
				\end{minipage}
				\begin{minipage}{0.5\textwidth}
					\begin{figure}[H]
						\centering
						\begin{circuitikz}
							\tikzstyle{every node}=[font=\normalsize]
							\draw [->, >=Stealth] (11.25,13) -- (14,13)node[pos=1,above]{Re};
							\draw [->, >=Stealth] (13.25,11) -- (13.25,15)node[pos=1,above]{Im};
							\node at (11.75,14.25) {$\times$};
							\node at (11.75,11.75) {$\times$};
							\draw [short] (13.25,13) -- (11.75,14.25)node[pos=0.5,above, sloped]{$\omega_n$};
							\draw [dashed] (11.75,14.25) -- (13.25,14.25)node[pos=1,right]{$j\omega_d$};
							\draw [dashed] (11.75,14.25) -- (11.75,11.75)node[pos=0.4,left]{$\sigma$};
							\draw [short] (11.75,11.75) -- (13.25,13);
							\draw [<->, >=Stealth] (12.5,13.6) .. controls (12.25,13.3) and (12.25,13.25) .. (12.25,13)node[pos=0.5,left]{$\theta$};
							\draw [ color={rgb,255:red,0; green,128; blue,255}, short] (11.75,11.75) -- (11.75,11);
							\draw [ color={rgb,255:red,0; green,128; blue,255}, short] (11.75,14.25) -- (11.75,15);
						\end{circuitikz}
						
						\label{fig:my_label}
					\end{figure}
				\end{minipage}
			\end{figure}
			
			Respuesta a escalón en sistemas subamortiguados ($0 <\xi < 0.707$):
			\[y(t) = K\left(1-\dfrac{e^{-\sigma t}}{\sqrt{1-\xi^2}} \sin (\omega_d t + \theta)\right)\]
			
			\begin{figure}[H]
				\begin{minipage}{0.5\textwidth}
					\begin{itemize}
						\item \textbf{\textit{Tiempo de establecimiento:}} tiempo que tarda en llegar al valor final con un error del 5\%: \[t_s = \dfrac{\pi}{\sigma}\]
						\item \textbf{\textit{Tiempo de pico:}} \[t_p = \dfrac{\pi}{\omega_d}\]
						\item \textbf{\textit{Sobreoscilación:}} \[M_p = e^{-\pi/\tan \theta}\cdot 100\%\]
						\item \textbf{\textit{Tiempo de subida:}} tiempo que tarda en alcanzar el 100\% desde el inicio del transitorio: \[t_r = \dfrac{\pi - \theta}{\omega_d}\]
					\end{itemize}
				\end{minipage}
				\begin{minipage}{0.5\textwidth}
					\begin{figure}[H]
						\centering
							\begin{circuitikz}[scale = 1.3]
								\tikzstyle{every node}=[font=\normalsize]
								\draw [->, >=Stealth] (13.75,12.75) -- (17.5,12.75)node[pos=1,above]{t};
								\draw [->, >=Stealth] (14,12.5) -- (14,16)node[pos=1,right]{y};
								\draw [ color={rgb,255:red,0; green,128; blue,255}, short] (14,12.75) .. controls (14.25,15) and (14.5,16) .. (15,14.75);
								\draw [dashed] (17.25,14.75) -- (14,14.75)node[pos=1,left]{K};
								\draw [ color={rgb,255:red,0; green,128; blue,255}, short] (15,14.75) .. controls (15.5,13.5) and (15.5,15.75) .. (16,14.75);
								\draw [ color={rgb,255:red,0; green,128; blue,255}, short] (16,14.75) .. controls (16.5,14) and (16.5,15.25) .. (16.75,14.75);
								\draw [dashed] (14.6,15.3) -- (14.6,12.75)node[pos=1,below]{$t_p$};
								\draw [dashed] (15.75,15) -- (15.75,12.75)node[pos=1,below]{$t_s$};
								\draw [dashed] (14.3,14.75) -- (14.3,12.75)node[pos=1,below]{$t_r$};
								\draw [dashed] (14.6,15.3) -- (14,15.3)node[pos=1,left]{$K \cdot M_p$};
							\end{circuitikz}
						
						\label{fig:my_label}
					\end{figure}
				\end{minipage}
			\end{figure}
			
			
		\subsection{Sistemas de tercer orden.}
			Constan de tres polos. Se distinguen, por simplificar, 2 casos fundamentales:
			\begin{itemize}
				\item \textbf{\textit{Tres polos reales:}}
				 	La salida de la función de transferencia tendrá 3 componentes de la forma:
				 	\[A_1 e^{s_1 t},\,A_2 e^{s_2 t},\,A_3 e^{s_3 t}\]
				 	
				 	Luego para que el sistema sea estable $s_1$, $s_2$ y $s_3$ deben ser negativos en su parte real.
				 	
				 \item \textbf{\textit{Un polo real y dos complejos conjugados:}}
				 	La salida de la función de transferencia tendrá la forma:
				 	\[A_1\cdot e^{\sigma t} \sin (\omega t + \beta)\]
				 	
				 	Para que la exponencial decrezca $\sigma$ debe ser negativa.
			\end{itemize}
	
	
	\begin{figure}[H]
		\centering
		\begin{circuitikz}
			\tikzstyle{every node}=[font=\normalsize]
			\draw [short] (7.25,10.25) -- (7.25,8.75);
			\draw [short] (6.25,9.25) -- (7.25,8.75);
			\draw [short] (6.25,9.75) -- (7.25,10.25);
			\node [font=\normalsize] at (3.5,10.75) {};
			\node [font=\normalsize] at (6.75,9.5) {TV};
			\draw [short] (7.25,9.75) -- (9,9.75);
			\draw [short] (7.25,9.25) -- (9,9.25);
			\draw [short] (9.75,10) -- (10,10);
			\draw [short] (10,10) -- (10,9.75);
			\draw [short] (9.75,9) -- (10,9);
			\draw [short] (10,9) -- (10,9.25);
			\draw [short] (10,9.25) -- (11.8,9.25);
			\draw [short] (10,9.75) .. controls (12.5,9.75) and (10.25,9.75) .. (11.8,9.75);
			\draw  (12.5,9.5) circle (0.75cm) node {\normalsize G} ;
			\draw [line width=0.2pt, short] (12.25,9.25) .. controls (12.5,9.5) and (12.5,9) .. (12.75,9.25);
			\draw [short] (13.25,9.5) -- (15.5,9.5);
			\draw [short] (14.75,9.5) -- (14.75,7);
			\draw  (11.5,7.5) rectangle (13.5,6.5);
			\node [font=\normalsize] at (12.5,7.25) {Regulador};
			\node [font=\normalsize] at (12.5,6.75) {AVR};
			\draw (11.75,8.25) to[L ] (13.25,8.25);
			\draw [short] (11.75,8.25) -- (11.75,7.5);
			\draw [short] (13.25,8.25) -- (13.25,7.5);
			\draw [line width=0.2pt, short] (13.5,9.25) -- (13.75,9.75);
			\draw [line width=0.2pt, short] (13.75,9.25) -- (14,9.75);
			\draw [line width=0.2pt, short] (14,9.25) -- (14.25,9.75);
			\draw [line width=0.2pt, short] (14.25,9.25) -- (14.5,9.75);
			\draw [short] (9.75,10) -- (9.75,11.25);
			\draw [short] (9.75,10) -- (9.5,10);
			\draw [short] (9.5,10) -- (9.5,9.75);
			\draw [short] (9.5,9.25) -- (9.5,9);
			\draw [short] (9.5,9) -- (9.75,9);
			\draw [short] (9,9.75) -- (9.5,9.75);
			\draw [short] (9,9.25) -- (9.5,9.25);
			\draw [short] (6.25,9.75) -- (6.25,9.25);
			\draw  (8.75,12.25) rectangle (10.75,11.25);
			\node [font=\normalsize] at (9.75,12) {Regulador};
			\node [font=\normalsize] at (9.75,11.5) {velocidad};
			\draw [->, >=Stealth] (8,10) .. controls (8.25,9.5) and (8.25,9.5) .. (8,9) ;
			\draw [->, >=Stealth] (8.75,10) .. controls (9,9.5) and (9,9.5) .. (8.75,9) ;
			\draw [->, >=Stealth] (11.25,9) .. controls (11,9.5) and (11,9.5) .. (11.25,10) ;
			\node [font=\normalsize] at (8,8.75) {n};
			\node [font=\normalsize] at (10,10.75) {n};
			\node [font=\normalsize] at (11.25,10.25) {$T_{elec}$};
			\node [font=\normalsize] at (8.75,8.75) {$T_{mec}$};
			\draw [->, >=Stealth] (9.75,10) -- (9.75,11.25);
			\draw [->, >=Stealth] (12,11.75) -- (10.75,11.75);
			\draw [->, >=Stealth] (12.5,5.75) -- (12.5,6.5);
			\draw [->, >=Stealth] (11.5,7.75) -- (11.5,8.25);
			\draw [->, >=Stealth] (14.75,7) -- (13.5,7);
			\node [font=\normalsize] at (14.75,8.25) {};
			\node [font=\normalsize] at (12.5,5.5) {$V_{ref}$};
			\node [font=\normalsize] at (11.25,8) {$I_{ex}$};
			\node [font=\normalsize] at (15,8) {V};
			\node [font=\normalsize] at (12.5,11.75) {$P_{ref}$};
			\draw  (8,11.75) circle (0.5cm);
			\draw [short] (8.75,11.75) -- (8.5,11.75);
			\node [font=\normalsize] at (8,11.75) {$M$};
			\draw [short] (7.5,11.75) -- (5.75,11.75);
			\draw [->, >=Stealth] (5.75,11.75) -- (5.75,11.25);
			\draw [short] (5.75,11.25) -- (6.25,11.5);
			\draw [short] (5.75,11.25) -- (6.25,11);
			\draw [short] (6.25,11) -- (6.25,11.5);
			\draw [short] (5.75,11.25) -- (5.25,11.5);
			\draw [short] (5.25,11.5) -- (5.25,11);
			\draw [short] (5.25,11) -- (5.75,11.25);
			\draw [short] (6.75,10) -- (6.75,11.25);
			\draw [short] (6.25,11.25) -- (6.75,11.25);
			\draw [short] (5.25,11.25) -- (4.75,11.25);
			\node [font=\normalsize] at (8,12.5) {$Servomotor$};
			\node [font=\normalsize] at (4,11.5) {$Entrada$};
			\node [font=\normalsize] at (4,11) {$vapor$};
			\draw [ color={rgb,255:red,2; green,141; blue,37} , dashed] (10.5,10.75) rectangle  (16,5);
			\draw [ color={rgb,255:red,234; green,72; blue,72}, dashed] (13,10.75) -- (13,13.5);
			\draw [ color={rgb,255:red,234; green,72; blue,72}, dashed] (13,13.5) -- (3.25,13.5);
			\draw [ color={rgb,255:red,234; green,72; blue,72}, dashed] (3.25,13.5) -- (3.25,7.75);
			\draw [ color={rgb,255:red,234; green,72; blue,72}, dashed] (3.25,7.75) -- (10.5,7.75);
			\node [font=\normalsize, color={rgb,255:red,234; green,72; blue,72}] at (4.25,13) {Control P-f};
			\node [font=\normalsize, color={rgb,255:red,62; green,167; blue,89}] at (14.75,5.5) {Control Q-V};
		\end{circuitikz}
		\label{fig:my_label}
	\end{figure}

