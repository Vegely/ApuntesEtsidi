\chapter{Cobertura de la demanda. Mercado eléctrico.}
\section{Explotación del mercado eléctrico.}
La energía eléctrica se genera en grandes concentradas de forma concentrada y posteriormente se transmite a grandes distancias donde el equilibrio se obtiene de las curvas de oferta y demanda.


Una misión importante de la red es la de ajustar la energía generada a la demandada con unos valores de tensión (Control Q-U) y frecuencia (Control P-f).
\section{Curva de demanda diaria.}
La demanda varia constantemente tanto hora a hora como diariamente. Esta demanda se refleja en las curvas de carga o demanda donde la diferencia entre ambas 


\section{Pérdidas en la red.}
g
\section{Gestión de la red.}
g
\subsection{Incidencias no previstas.}
g
\subsection{Configuración sistema eléctrico de potencia.}
g
\section{Funcionamiento del mercado eléctrico.}
g
\subsection{Actividades principales.}
g
\subsection{Reparto de la distribución de energía eléctrica.}
g
\subsection{Comercialización.}
g
\subsection{Organización.}
g
\subsection{Instituciones reguladoras.}
g
\subsection{Operador del mercado (OMIE).}
g
\subsection{Operador del sistema (REE).}
g
\section{Mercado ibérico (MIBEL).}
g
\section{Interconexiones con el extranjero.}
g
\subsection{Francia.}
g
\subsection{Portugal.}
g
\subsection{Marruecos.}
g
\subsection{Gestión de las interconexiones.}
g
\section{Mercado intradiario.}
g
\subsection{Secuencia de los procesos del mercado.}
g
\subsection{Mercados a plazo.}
g
\subsection{Mercado organizado diario (casación horaria).}
g
\subsection{Tipos de oferta de venta de energía.}
g
\subsection{Proceso de casación.}
g
\subsection{Curva de oferta de venta de energía.}
g
\subsection{Influencia fuentes renovables.}
g
\subsection{Retribuciones para amortizar costes fijos.}
g
\subsection{Curva de demanda.}
g
\section{Mercado de restricciones técnicas.}
g
\section{Mercado de servicios complementarios.}
g
\subsection{Regulación primaria.}
g
\subsection{Regulación secundaria.}
g
\subsection{Control de tensión.}
g
\subsection{Reservas de potencia.}
g
\subsection{Gestión de desvíos.}
g
\subsection{Regulación terciaria.}
g
\section{Precio medio final.}
g
\subsection{Costes recogidos en la tarifa eléctrica (PVPC).}
g
\section{Programación de la generación de electricidad.}
g
\subsection{Curva acumulada de demanda anual.}
g
\subsection{Curva de demanda anual.}
g
\subsection{Parámetros principales curva de demanda.}
g
\subsection{Curva acumulada de generación anual.}
g
\subsection{Parámetros principales curva de generación.}
g
\section{Reserva de potencia.}
g
\subsection{Características estáticas.}
g
\subsection{Características dinámicas.}
g
\subsection{Secuenciamiento óptimo de grupos.}
g
\section{Costes de generación.}
g
\subsection{Comparativa de costes.}
g
\subsection{Coste de inversión o fijo.}
g
\subsection{Costes variables.}
g
\subsection{Coste total.}
g
\section{Aspectos técnicos de la producción de energía.}
g