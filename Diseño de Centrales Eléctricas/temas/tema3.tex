\chapter{Tecnologías más eficientes. Captura de $\textbf{CO}_2$}
\section{Evolución de las calderas de combustión de carbón.}

\subsection{Turbina de vapor.}

\section{Tecnología de captura y almacenamiento de CO$_2$.}

\subsection{Combustión convencional del carbón pulverizado.}

\subsection{Co-combustión.}

\subsection{Captura CO$_2$ antes de la combustión.}

\subsection{Captura CO$_2$ durante la combustión.}

\subsection{Captura CO$_2$ tras la combustión.}

\subsection{Transporte y almacenamiento de CO$_2$.}

\subsection{Transformación CO$_2$ en metano.}

\section{Calderas de combustión de lecho fluidizado (CLF).}

\subsection{Ventajas CLF.}

\subsection{Tipos de CLF.}

\section{Gasificación del carbón.}

\subsection{Tipos de gasificadores.}

\subsection{Estado actual de la gasificación.}

\section{Hidrógeno.}

\subsection{Producción del hidrógeno.}

\subsection{Clasificación del hidrógeno.}

\subsection{Almacenamiento del hidrógeno.}

\subsection{Aplicaciones del hidrógeno.}

\section{Ciclos combinados.}

\section{Fusión nuclear.}

\section{Almacenamiento de energía.}

\subsection{Volantes de inercia.}
g
\subsection{Supercondensadores.}
g
\subsection{Baterías.}
g
\subsection{Aire comprimido (CAES).}
h
\subsection{Centrales de bombeo.}