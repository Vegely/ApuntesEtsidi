\section{Tema 4. Medidas de tensiones, intensidades y resistencias.}
\subsection{Métodos industriales frente a métodos de laboratorio.}
\begin{itemize}
	\item \textbf{Métodos industriales:}
	\begin{itemize}
		\item Utilizan aparatos de medida básicos.
		\item No es necesaria formación especial.
		\item No son muy exactos, pero tienen precisión suficiente.
		\item Dan respuestas inmediatas.
	\end{itemize}
	\item \textbf{Métodos de laboratorio:}
	\begin{itemize}
		\item Requieren equipos de gran calidad metrológica.
		\item Los operadores deben ser capacitados para su uso.
	\end{itemize}
\end{itemize}


\subsection{Métodos industriales para medir tensión e intensidad.}

\subsection{Error de inserción.}
\subsection{Medida de resistencia con óhmetros.}
\subsection{Medida de resistencia con voltímetro y amperímetro.}
\subsection{Métodos de laboratorio para medir tensión e intensidad.}
\subsection{Puente de Wheatstone para medida de resistencia.}
\subsection{Puente de Kelvin-Thomson.}