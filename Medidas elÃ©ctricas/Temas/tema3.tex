\section{Tema 3. Adaptadores y convertidores de medida.}
	\subsection{Divisores de tensión e intensidad.}
	\textbf{Necesidad de adaptar las medidas:} los alcances de los aparatos de medida están limitados a determinados valores, unas veces por seguridad y otras por razones de diseño o constructivas. Cuando las magnitudes a medir superan los campos de los aparatos de medida hay que realizar una adaptación de las mismas.
	
		\subsubsection{Divisores de tensión resistivos.}
			\begin{itemize}
				\item[-] Ampliamente utilizados en todo tipo de circuitos eléctricos y electrónicos de B.T.
				\item[-] Alcances de $<1\,kV$ y $<100\,A$.
				\item[-] Utilizados para ampliar el C.M. de voltímetros y amperímetros, en c.c. y c.a.
				\item[-] Buena respuesta en frecuencia. No se ven afectados por la forma de onda.
				\item[-] Su error sistemático de inserción es el mismo que el de un aparato equivalente con igual resistencia interna.
				\item[-] Una vez construidos deben contrastarse y reclasificar los equipos de acuerdo con las nuevas incertidumbres.
			\end{itemize}
			
			Configuración básica:
			\begin{figure}[H]
				\centering
					\begin{circuitikz}
						\tikzstyle{every node}=[font=\large]
						\draw (6.5,20) to[R] (6.5,18);
						\draw (6.5,18) to[R] (6.5,16);
						\draw (6.5,18) to[short, -*] (6.5,18);
						\node [font=\large] at (7,19) {$R_S$};
						\node [font=\large] at (7,17) {$R_P$};
						\draw (9.5,18) to[R] (9.5,16);
						\node [font=\large] at (10,17) {$R_V$};
						\draw (6.5,18) to[short, -*] (8.25,18);
						\draw (6.5,16) to[short, -*] (8.25,16);
						\draw (6.5,16) to[short, -*] (6.5,16);
						\draw (6.5,16) to[short, -*] (6.5,16);
						\draw (6.5,16) to[short, -o] (5.25,16);
						\draw (6.5,20) to[short, -o] (5.25,20);
						\node [font=\large] at (8.6,17) {$U_2$};
						\draw (8.25,18) to[short] (9.5,18);
						\draw (8.25,16) to[short] (9.5,16);
						\node [font=\large] at (4.9,18) {$U_1$};
						\draw [-latex] (5.25,19.75) -- (5.25,16.25);
						\draw [-latex] (8.25,17.75) -- (8.25,16.25);
					\end{circuitikz}
			\end{figure}
			
			$R_V$ ha de ser $>100\cdot R_P$ para que el error de inserción $\varepsilon_I<1\%$.
			Para reducir la disipación de potencia $R_S$ y $R_P$ tienen que ser de valor elevado.
			
			
			Se suele utilizar en circuitos electrónicos con tensiones reducidas ($<100\,V$).
			
			\subsubsection*{Ampliación del campo de medida de un voltímetro.}
			En el siguiente esquema se aplica tensión en alguna de las 3 bornas $V_1$, $V_2$ o $V_3$.
			\begin{figure}[H]
				\centering
					\begin{circuitikz}
						\tikzstyle{every node}=[font=\large]
						\draw (2.75,16.75) to[R] (2.75,14.5);
						\draw (4.5,16.75) to[R] (4.5,14.5);
						\draw (6.25,16.75) to[R] (6.25,14.5);
						\node [font=\large] at (2.25,15.6) {$R_3$};
						\node [font=\large] at (4,15.6) {$R_2$};
						\node [font=\large] at (5.75,15.6) {$R_1$};
						\draw (7.5,16.75) to[R] (9.5,16.75);
						\node [font=\large] at (8.5,17.25) {$R_m$};
						\node [font=\large] at (8.5,16) {$V_m$};
						\draw [-latex] (7.75,16.25) -- (9.25,16.25);
						\draw (2.75,16.75) to[short] (7.5,16.75);
						\draw (9.75,14.25) to[short, o-] (9.75,16.75);
						\draw (6.25,14.25) to[short, o-] (6.25,14.5);
						\draw (4.5,14.25) to[short, o-] (4.5,14.5);
						\draw (2.75,14.25) to[short, o-] (2.75,14.5);
						\node [font=\large] at (2.75,13.75) {$V_3$};
						\node [font=\large] at (4.5,13.75) {$V_2$};
						\node [font=\large] at (6.25,13.75) {$V_1$};
						\node [font=\large] at (9.75,13.75) {(-)};
						\draw  (8.5,16.75) circle (1cm);
						\draw (9.5,16.75) to[short] (9.75,16.75);
						\draw (6.25,16.75) to[short, -*] (6.25,16.75);
						\draw (4.5,16.75) to[short, -*] (4.5,16.75);
						\draw [-latex] (6.25,16.75) -- (7,16.75)node[pos=0.9,above]{I};
					\end{circuitikz}
				\label{fig:my_label}
			\end{figure}
			
			Se define el \textit{"factor amplificador"} como:
			\[n_i=\dfrac{V_i}{V_m} / i \in \mathbb{N}\]
			\[R_i=R_m\cdot (n_i-1)\]
			\[R_{V_i}=R_i+R_m\]
			
			
			Normalmente $V_i >\!> V_m$, luego \[P_{R_i} \approx \dfrac{V_i^2}{R_i}\]
			
		\subsubsection{Divisores de tensión inductivos.}
			\begin{figure}[H]
				\centering
					\begin{circuitikz}
						\tikzstyle{every node}=[font=\large]
						\draw (5.25,19.5) to[L,l={ \large $R_SX_S$} ] (5.25,18);
						\draw (5.25,18) to[L,l={ \large $R_PX_P$} ] (5.25,16.5);
						\draw (5.25,19.75) to[short, -o] (4.5,19.75);
						\draw (5.25,16.25) to[short, -o] (4.5,16.25);
						\draw (5.25,19.75) to[short] (5.25,19.5);
						\draw (5.25,16.5) to[short] (5.25,16.25);
						\draw (5.25,16.25) to[short, -*] (7.5,16.25);
						\draw [-latex] (6.25,18) -- (5.25,18);
						\draw (6.25,18) to[short, -*] (7.5,18);
						\draw [-latex] (4.5,19.5) .. controls (4.5,17.75) and (4.5,18) .. (4.5,16.5) node[pos=0.4,left]{$U_1$};
						\draw [-latex] (7.5,17.75) -- (7.5,16.5)node[pos=0.5,right]{$U_2$};
						\draw [-] (9,18) -- (7.5,18);
						\draw [-] (9,16.25) -- (7.5,16.25);
						\draw (9,18) to[R,l={ \large $R_C$}] (9,16.25);
					\end{circuitikz}
			\end{figure}
		
		\subsubsection{Divisores de tensión capacitivos.}
			\begin{figure}[H]
				\centering
					\begin{circuitikz}
						\tikzstyle{every node}=[font=\large]
						\draw (4.25,17.25) to[C,l={ \large $C_S$}] (4.25,15.75);
						\draw (4.25,15.75) to[C,l={ \large $C_P$}] (4.25,14.25);
						\draw  (7.5,15) circle (0.5cm) node {\large V} ;
						\draw (4.25,15.75) to[short] (7.5,15.75);
						\draw (7.5,15.75) to[short] (7.5,15.5);
						\draw (4.25,14.25) to[short] (7.5,14.25);
						\draw (7.5,14.25) to[short] (7.5,14.5);
						\draw (4.25,15.75) to[short, -*] (4.25,15.75);
						\draw (4.25,14.25) to[short, -*] (4.25,14.25);
						\draw (4.25,14.25) to[short, -o] (3,14.25);
						\draw (4.25,17.25) to[short, -o] (3,17.25);
						\draw [-latex] (3,17) -- (3,14.5)node[pos=0.5,left]{$U_1$};
						\draw (5.75,15.75) to[short, -*] (5.75,15.75);
						\draw (5.75,14.25) to[short, -*] (5.75,14.25);
						\draw [-latex] (5.75,15.5) -- (5.75,14.5)node[pos=0.5,right]{$U_2$};
					\end{circuitikz}
			\end{figure}
		
		\subsubsection{Divisores de intensidad resistivos.}
			\begin{figure}[H]
				\centering
					\begin{circuitikz}
						\tikzstyle{every node}=[font=\large]
						\draw (4.75,15.75) to[R,l={ \large $R_P$}] (6.75,15.75);
						\draw (4.75,17.5) to[R,l={ \large $R_m$}] (6.75,17.5);
						\draw  (5.75,17.5) circle (0.75cm);
						\draw [-latex] (4.25,17.5) -- (4.75,17.5)node[pos=0.6,above]{$I_m$};
						\draw (4.25,17.5) to[short] (4.25,15.75);
						\draw [-latex] (3.25,15.75) -- (3.75,15.75)node[pos=0.75,above]{I};
						\draw (3.75,15.75) to[short] (4.25,15.75);
						\draw (4.25,15.75) to[short, -*] (4.25,15.75);
						\draw (3.25,15.75) to[short, -o] (3,15.75);
						\draw [-latex] (4.25,15.75) -- (4.75,15.75)node[pos=0.9,above]{$I_d$};
						\draw (6.75,17.5) to[short] (7.25,17.5);
						\draw (7.25,17.5) to[short] (7.25,15.75);
						\draw (6.75,15.75) to[short] (8,15.75);
						\draw (8,15.75) to[short, -o] (8.25,15.75);
						\draw (7.25,15.75) to[short, -*] (7.25,15.75);
					\end{circuitikz}
			\end{figure}
			
		\subsubsection*{Shunt de Ayrton.}
			\begin{figure}[H]
				\centering
					\begin{circuitikz}
						\tikzstyle{every node}=[font=\large]
						\draw [line width=0.2pt, short] (8.75,15.5) -- (8.75,15)node[pos=1.75,below]{(-)};
						\draw [line width=0.2pt, short] (6.75,15.5) -- (6.75,15)node[pos=1.75,below]{$I_3$};
						\draw [line width=0.2pt, short] (4.75,15.5) -- (4.75,15)node[pos=1.75,below]{$I_2$};
						\draw [line width=0.2pt, short] (2.75,15.5) -- (2.75,15)node[pos=1.75,below]{$I_1$};
						\draw (4.75,17.5) to[R,l={ \large $R_m$}] (6.75,17.5);
						\draw  (5.75,17.5) circle (0.75cm);
						\draw [-latex] (4.25,17.5) -- (4.75,17.5)node[pos=0.6,above]{$I_m$};
						\draw (6.75,17.5) to[short] (7.25,17.5);
						\draw (2.75,15.75) to[R,l={ \large $R_1$}] (4.75,15.75);
						\draw (4.75,15.75) to[R,l={ \large $R_2$}] (6.75,15.75);
						\draw (6.75,15.75) to[R,l={ \large $R_3$}] (8.75,15.75);
						\draw (4.25,17.5) to[short] (2.75,17.5);
						\draw (2.75,17.5) to[short] (2.75,15.75);
						\draw (7.25,17.5) to[short] (8.75,17.5);
						\draw (8.75,17.5) to[short] (8.75,15.75);
						\draw (2.75,14.75) to[short, o-] (2.75,15.75);
						\draw (4.75,14.75) to[short, o-] (4.75,15.75);
						\draw (6.75,14.75) to[short, o-] (6.75,15.75);
						\draw (8.75,14.75) to[short, o-] (8.75,15.75);
						\draw (2.75,15.75) to[short, -*] (2.75,15.75);
						\draw (4.75,15.75) to[short, -*] (4.75,15.75);
						\draw (6.75,15.75) to[short, -*] (6.75,15.75);
						\draw (8.75,15.75) to[short, -*] (8.75,15.75);
					\end{circuitikz}
			\end{figure}
			
		\subsubsection*{Otros acoplamientos de resistencias.}
			\begin{figure}[H]
				\centering
					\begin{circuitikz}
						\tikzstyle{every node}=[font=\large]
						\draw (4.75,19.5) to[R,l={ \large $R_m$}] (6.75,19.5);
						\draw  (5.75,19.5) circle (0.75cm);
						\draw [-latex] (4.25,19.5) -- (4.75,19.5)node[pos=0.6,above]{$I_m$};
						\draw (1.75,17) to[R,l={ \large $R_1$}] (3.75,17);
						\draw (3.75,17) to[R,l={ \large $R_2$}] (5.75,17);
						\draw (5.75,17) to[R,l={ \large $R_3$}] (7.75,17);
						\draw (1.75,17) to[short, -*] (1.75,17);
						\draw (3.75,17) to[short, -*] (3.75,17);
						\draw (5.75,17) to[short, -*] (5.75,17);
						\draw (7.75,17) to[short, -*] (7.75,17);
						\draw (4.25,19.5) to[short] (3.75,19.5);
						\draw (3.75,19) to[short, o-] (3.75,19.5);
						\draw (2.75,18.25) to[short, -o] (3.25,18.25);
						\draw (4.75,18.25) to[short, -o] (4.25,18.25);
						\draw (3.75,17.75) to[short, -o] (3.75,18);
						\draw [-latex] (3.75,19) -- (3.25,18.25);
						\draw [-latex, dashed] (3.75,19) -- (3.75,18);
						\draw [-latex, dashed] (3.75,19) -- (4.25,18.25);
						\node [font=\small] at (3.25,18.5) {1};
						\node [font=\small] at (4,17.75) {2};
						\node [font=\small] at (4.5,18.5) {3};
						\draw (3.75,17.75) to[short] (3.75,17);
						\draw (4.75,18.25) to[short] (5.75,18.25);
						\draw (5.75,18.25) to[short] (5.75,17);
						\draw (6.75,19.5) to[short] (7.75,19.5);
						\draw (7.75,19.5) to[short] (7.75,17);
						\draw (2.75,18.25) to[short] (1.75,18.25);
						\draw (1.75,18.25) to[short] (1.75,17);
						\draw [-latex] (0.5,17) -- (1.25,17)node[pos=0.85,above]{$I$};
						\draw (1.25,17) to[short] (1.75,17);
						\draw (0.5,17) to[short, -o] (0.25,17);
						\draw (7.75,17) to[short, -o] (8.5,17);
						\node [font=\large] at (9,17) {(-)};
						\node [font=\large] at (-0.25,17) {(+)};
					\end{circuitikz}
			\end{figure}
			
	\subsection{Transformadores de medida.}
		\subsubsection{Utilidad de los trafos de medida.}
			\begin{figure}[!ht]
				\centering
				\resizebox{1\textwidth}{!}{%
					\begin{circuitikz}
						\tikzstyle{every node}=[font=\large]
						\draw [ ](3,21.25) to[L ] (3,19.25);
						\draw [ ](3.75,19.25) to[L ] (3.75,21.25);
						\draw [,  ](3.3,20.75) to[short] (3.3,19.75);
						\draw [,  ](3.45,20.75) to[short] (3.45,19.75);
						\draw [,   ] (5,20.25) circle (0.5cm) node {\large V} ;
						\draw [,  ](3.75,21.25) to[short] (5,21.25);
						\draw [,  ](5,21.25) to[short] (5,20.75);
						\draw [,  ](3.75,19.25) to[short] (5,19.25);
						\draw [,  ](5,19.75) to[short] (5,19.25);
						\draw [,  ](0.75,22.75) to[short] (0.75,18.5);
						\draw [,  ](2,22.75) to[short] (2,18.5);
						\draw[,  ] (3,21.25) to[short] (0.75,21.25);
						\draw[,  ] (3,19.25) to[short] (2,19.25);
						\draw (2,19.25) to[short, -*] (2,19.25);
						\draw (0.75,21.25) to[short, -*] (0.75,21.25);
						\node [font=\large] at (1.25,21.5) {A.T.};
						\node [font=\large] at (4.25,21.5) {B.T.};
						\draw [ ](5,19.25) to (5,19) node[ground]{};
						\draw (5,19.25) to[short, -*] (5,19.25);
						\draw [,  ](7.5,19.25) to[R,l={ \large $Z$}] (9.75,19.25);
						\draw [,  ](7.5,19.25) to[short] (7.5,22.5);
						\draw [,  ](9.75,21.25) to[short] (9.75,22.5);
						\draw [ ](9.75,21.25) to[L ] (9.75,20.25);
						\draw [,  ](10.5,21.75) to[short] (11.5,21.75);
						\draw [,  ](10.5,19.75) to[short] (11.5,19.75);
						\draw [,   ] (12.5,21.75) circle (0.5cm) node {\large A} ;
						\draw [,  ](11.5,21.75) to[short] (12,21.75);
						\draw [,  ](13,21.75) to[short] (13.5,21.75);
						\draw [,  ](11.5,19.75) to[short] (13.5,19.75);
						\draw [,  ](13.5,21.75) to[short] (13.5,19.75);
						\draw [-latex] (9.75,22) -- (9.75,21.5)node[pos=0.5,left]{$I_1$};
						\draw [-latex] (11.25,21.75) -- (11.5,21.75)node[pos=0.5,above]{$I_2$};
						\draw [,  ](9.75,20.25) to[short] (9.75,19.25);
						\draw [ ](10.5,20.25) to[L ] (10.5,21.25);
						\draw [,  ](10.05,21.25) to[short] (10.05,20.25);
						\draw [,  ](10.2,21.25) to[short] (10.2,20.25);
						\draw [,  ](10.5,21.75) to[short] (10.5,21.25);
						\draw [,  ](10.5,20.25) to[short] (10.5,19.75);
					\end{circuitikz}
				}%
				\label{fig:my_label}
			\end{figure}
		
		\subsubsection{Transformadores de tensión.}
			\begin{figure}[H]
				\centering
					\begin{circuitikz}
						\tikzstyle{every node}=[font=\large]
						\draw [-latex] (6.25,23.75) -- (5.25,25.25)node[pos=0.5,sloped,above]{$X_1·I_1$};
						\draw [-latex, dashed] (4.25,17.75) -- (5.25,20)node[pos=1,above]{$-I'_2$};
						\draw [-latex] (4.25,17.75) -- (7.75,17.75)node[pos=1,right]{$\Phi$};
						\draw [-latex] (4.25,17.75) -- (5.25,18.25)node[pos=1,right]{$I_0$};
						\draw [ color={rgb,255:red,0; green,0; blue,255}, -latex] (4.25,17.75) -- (6.25,20.5)node[pos=1,right]{$I_1$};
						\draw [dashed] (5.25,18.25) -- (6.25,20.5);
						\draw [dashed] (5.25,20) -- (6.25,20.5);
						\draw [ color={rgb,255:red,128; green,0; blue,255}, -latex] (4.25,17.75) -- (5.25,25.25)node[pos=1,left]{$U_1$};
						\draw [-latex, dashed] (4.25,17.75) -- (4.25,22.5)node[pos=1,left]{$E_1$};
						\draw [-latex] (4.25,22.5) -- (6.25,23.75)node[pos=0.7,sloped,above]{$R_1·I_1$};
						\draw [-latex, dashed] (4.25,17.75) -- (3,22.25)node[pos=1,left]{$-K_U·U_2$};
						\draw [-latex] (3.25,21.25) .. controls (3.75,21.75) and (4.25,21.75) .. (4.75,21.5)node[pos=0.3,above]{$\delta_U$};
						\draw [ color={rgb,255:red,255; green,0; blue,0}, -latex] (4.25,17.75) -- (5,15.5)node[pos=0.4,right]{$U_2$};
						\draw [ color={rgb,255:red,255; green,0; blue,0}, -latex] (5,15.5) -- (3.5,13.75)node[pos=0.5,sloped,above]{$R_2·I_2$};
						\draw [ color={rgb,255:red,255; green,0; blue,0}, -latex] (3.5,13.75) -- (4.25,13)node[pos=0.5,sloped,below]{$X_2·I_2$};
						\draw [-latex] (4.25,17.75) -- (4.25,13)node[pos=1,right]{$E_2$};
						\draw [ color={rgb,255:red,0; green,217; blue,0}, -latex] (4.25,17.75) -- (2,15.25)node[pos=1,left]{$I_2$};
						\draw [-latex] (3.25,16.5) .. controls (3.5,16) and (4.25,16) .. (4.75,16.25)node[pos=0.2,below]{$\varphi_2$};
					\end{circuitikz}
			\end{figure}
			
		\subsubsection{Transformadores de intensidad.}
			\begin{figure}[H]
				\centering
					\begin{circuitikz}
						\tikzstyle{every node}=[font=\large]
						\draw [, -latex, dashed] (3,15) -- (5.25,20)node[pos=1,above]{$-I'_2$};
						\draw [, -latex] (4.25,17.75) -- (7.75,17.75)node[pos=1,right]{$\Phi$};
						\draw [, -latex] (4.25,17.75) -- (5.25,18.25)node[pos=1,right]{$I_0$};
						\draw [ color={rgb,255:red,0; green,0; blue,255}, , -latex] (4.25,17.75) -- (6.25,20.5)node[pos=1,right]{$I_1$};
						\draw [, dashed] (5.25,18.25) -- (6.25,20.5);
						\draw [, dashed] (5.25,20) -- (6.25,20.5);
						\draw [, -latex, dashed] (4.25,17.75) -- (4.25,22.5)node[pos=1,above]{$E_1$};
						\draw [, -latex, dashed] (4.25,17.75) .. controls (3.75,20) and (3.75,20) .. (3,22.25)node[pos=1]{$-K_U·U_2$};
						\draw [, -latex] (4.25,17.75) -- (4.25,13)node[pos=1,right]{$U_2$};
						\draw [ color={rgb,255:red,0; green,217; blue,0}, , -latex] (4.25,17.75) -- (2,15.25)node[pos=1,left]{$I_2$};
						\draw [, -latex] (2,16.5) -- (2.5,16);
						\draw [, -latex] (4,15.25) -- (3.25,15.5);
						\draw [, dashed] (2.5,16) -- (3.25,15.5)node[pos=0.2,below]{$\delta_i$};
					\end{circuitikz}
			\end{figure}
			
			\begin{figure}[H]
				\centering
					\begin{circuitikz}
						\tikzstyle{every node}=[font=\large]
						\node  [waves, right, rotate=180, ] at(11.75,15) {};
						\draw [, dashed] (3.25,6) -- (5.25,10.5)node[pos=0.5,above]{$\varepsilon_I$};
						\draw [, -, dashed] (1.25,5) -- (3.25,9.5)node[pos=1,above]{$K_I·I_2$};
						\draw [ color={rgb,255:red,0; green,0; blue,255}, , -] (1.25,5) -- (5.25,10.5)node[pos=1,right]{$I_1$};
						\draw [, -latex] (1.25,5) -- (3.25,6)node[pos=1,right]{$I_0$};
						\draw [, -latex] (1.25,5) -- (4.75,5);
						\draw [, -latex, dashed] (1.25,5) -- (1.25,10.25)node[pos=1,above]{$E_1$};
						\draw [, dashed] (3.25,9.5) -- (5.25,10.5)node[pos=0.5,above]{$\vec{\varepsilon_I}$};
						\draw [, short] (2.5,7.75) -- (3,7.5)node[pos=0.9,above]{$\delta_I$};
						\draw [, -latex] (2,8) -- (2.5,7.75);
						\draw [, -latex] (3.5,7.25) -- (3,7.5);
						\draw [, dashed] (3.25,9.5) -- (4.5,9);
						\draw [, short] (3.75,9.5) -- (4.75,10);
						\draw [, short] (4.75,10) -- (4.5,9.25);
						\draw [, short] (3.75,9.5) -- (4.5,9.25);
					\end{circuitikz}
			\end{figure}
			
			\begin{figure}[!ht]
				\centering
					\begin{circuitikz}
						\tikzstyle{every node}=[font=\large]
						\draw [, line width=0.5pt ] (1.75,4.75) circle (0.5cm) node {\large W} ;
						\draw [line width=0.5pt](1,3.25) to[L ] (2.5,3.25);
						\draw [line width=0.5pt](1,2.5) to[L ] (2.5,2.5);
						\draw [, line width=0.5pt](1.25,3) to[short] (2.25,3);
						\draw [, line width=0.5pt](1.25,2.75) to[short] (2.25,2.75);
						\draw [, line width=0.5pt](1.75,4.25) to[short] (1.75,3.75);
						\draw [, line width=0.5pt](1.75,3.75) to[short] (2.75,3.75);
						\draw [, line width=0.5pt](2.75,3.75) to[short] (2.75,3.25);
						\draw [, line width=0.5pt](2.75,3.25) to[short] (2.75,2.5);
						\draw [, line width=0.5pt](1.75,5.25) to[short] (1.75,5.75);
						\draw[, line width=0.5pt] (1.75,5.75) to[short] (0,5.75);
						\draw [, line width=0.5pt](0,5.75) to[short] (0,3.75);
						\draw [, line width=0.5pt](2.75,2.75) to[short] (2.75,1);
						\draw [, line width=0.5pt](-1,1) to[short] (3.5,1);
						\draw (2.75,1) to[short, -*] (2.75,1);
						\draw[, line width=0.5pt] (1.25,4.75) to[short] (0.5,4.75);
						\draw [, line width=0.5pt](0.5,4.75) to[short] (0.5,3.25);
						\draw [, line width=0.5pt](0.5,3.25) to[short] (1,3.25);
						\draw [, line width=0.5pt](2.25,4.75) to[short] (3,4.75);
						\draw [, line width=0.5pt](3,4.75) to[short] (3,3.25);
						\draw [, line width=0.5pt](2.75,3.25) to[short] (3,3.25);
						\draw [, line width=0.5pt](2.5,3.25) to[short] (2.75,3.25);
						\draw [, line width=0.5pt](-1,2.5) to[short] (1,2.5);
						\draw [, line width=0.5pt](2.5,2.5) to[short] (3.5,2.5);
						\draw [, line width=0.5pt](0,3.75) to[short] (0,2.5);
						\draw (0,2.5) to[short, -*] (0,2.5);
						\draw [, line width=0.5pt](3.5,2.5) to[R,l={ \large $Z$}] (3.5,1);
						\draw [line width=0.5pt, -latex] (0.5,3.5) -- (0.5,4.25)node[pos=0.9,right]{$I_2$};
						\draw [line width=0.5pt, -latex] (-1,2.5) -- (-0.5,2.5)node[pos=0.9,above]{$I_1$};
					\end{circuitikz}
			\end{figure}
			
	\subsection{Sensores de efecto Hall.}
		\subsubsection{Aplicaciones y fundamento.}
		\subsubsection{Cuestiones sobre sensores de efecto Hall.}
		\subsubsection{Ejemplos de características y aplicaciones.}