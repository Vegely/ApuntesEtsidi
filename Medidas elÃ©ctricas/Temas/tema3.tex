\section{Tema 3. Adaptadores y convertidores de medida.}
	\subsection{Divisores de tensión e intensidad.}
	\textbf{Necesidad de adaptar las medidas:} los alcances de los aparatos de medida están limitados a determinados valores, unas veces por seguridad y otras por razones de diseño o constructivas. Cuando las magnitudes a medir superan los campos de los aparatos de medida hay que realizar una adaptación de las mismas.
	
		\subsubsection{Divisores de tensión resistivos.}
			\begin{itemize}
				\item[-] Ampliamente utilizados en todo tipo de circuitos eléctricos y electrónicos de B.T.
				\item[-] Alcances de $<1\,kV$ y $<100\,A$.
				\item[-] Utilizados para ampliar el C.M. de voltímetros y amperímetros, en c.c. y c.a.
				\item[-] Buena respuesta en frecuencia. No se ven afectados por la forma de onda.
				\item[-] Su error sistemático de inserción es el mismo que el de un aparato equivalente con igual resistencia interna.
				\item[-] Una vez construidos deben contrastarse y reclasificar los equipos de acuerdo con las nuevas incertidumbres.
			\end{itemize}
			
			Configuración básica:
			\begin{figure}[H]
				\centering
					\begin{circuitikz}
						\tikzstyle{every node}=[font=\large]
						\draw (6.5,20) to[R] (6.5,18);
						\draw (6.5,18) to[R] (6.5,16);
						\draw (6.5,18) to[short, -*] (6.5,18);
						\node [font=\large] at (7,19) {$R_S$};
						\node [font=\large] at (7,17) {$R_P$};
						\draw (9.5,18) to[R] (9.5,16);
						\node [font=\large] at (10,17) {$R_V$};
						\draw (6.5,18) to[short, -*] (8.25,18);
						\draw (6.5,16) to[short, -*] (8.25,16);
						\draw (6.5,16) to[short, -*] (6.5,16);
						\draw (6.5,16) to[short, -*] (6.5,16);
						\draw (6.5,16) to[short, -o] (5.25,16);
						\draw (6.5,20) to[short, -o] (5.25,20);
						\node [font=\large] at (8.6,17) {$U_2$};
						\draw (8.25,18) to[short] (9.5,18);
						\draw (8.25,16) to[short] (9.5,16);
						\node [font=\large] at (4.9,18) {$U_1$};
						\draw [-latex] (5.25,19.75) -- (5.25,16.25);
						\draw [-latex] (8.25,17.75) -- (8.25,16.25);
					\end{circuitikz}
			\end{figure}
			
			$R_V$ ha de ser $>100\cdot R_P$ para que el error de inserción $\varepsilon_I<1\%$.
			Para reducir la disipación de potencia $R_S$ y $R_P$ tienen que ser de valor elevado.
			
			
			Se suele utilizar en circuitos electrónicos con tensiones reducidas ($<100\,V$).
			
			\subsubsection*{Ampliación del campo de medida de un voltímetro.}
			En el esquema se aplica tensión en alguna de las 3 bornas $V_1$, $V_2$ o $V_3$.
			\begin{figure}[H]
				\centering
					\begin{circuitikz}
						\tikzstyle{every node}=[font=\large]
						\draw (2.75,16.75) to[R] (2.75,14.5);
						\draw (4.5,16.75) to[R] (4.5,14.5);
						\draw (6.25,16.75) to[R] (6.25,14.5);
						\node [font=\large] at (2.25,15.6) {$R_3$};
						\node [font=\large] at (4,15.6) {$R_2$};
						\node [font=\large] at (5.75,15.6) {$R_1$};
						\draw (7.5,16.75) to[R] (9.5,16.75);
						\node [font=\large] at (8.5,17.25) {$R_m$};
						\node [font=\large] at (8.5,16) {$V_m$};
						\draw [-latex] (7.75,16.25) -- (9.25,16.25);
						\draw (2.75,16.75) to[short] (7.5,16.75);
						\draw (9.75,14.25) to[short, o-] (9.75,16.75);
						\draw (6.25,14.25) to[short, o-] (6.25,14.5);
						\draw (4.5,14.25) to[short, o-] (4.5,14.5);
						\draw (2.75,14.25) to[short, o-] (2.75,14.5);
						\node [font=\large] at (2.75,13.75) {$V_3$};
						\node [font=\large] at (4.5,13.75) {$V_2$};
						\node [font=\large] at (6.25,13.75) {$V_1$};
						\node [font=\large] at (9.75,13.75) {(-)};
						\draw  (8.5,16.75) circle (1cm);
						\draw (9.5,16.75) to[short] (9.75,16.75);
						\draw (6.25,16.75) to[short, -*] (6.25,16.75);
						\draw (4.5,16.75) to[short, -*] (4.5,16.75);
						\draw [-latex] (6.25,16.75) -- (7,16.75)node[pos=0.9,above]{I};
					\end{circuitikz}
			\end{figure}
			
			Se define el \textit{"factor amplificador"} como:
			\[n_i=\dfrac{V_i}{V_m} / i \in \mathbb{N}\]
			\[R_i=R_m\cdot (n_i-1)\]
			\[R_{V_i}=R_i+R_m\]
			
			
			Normalmente $V_i >\!> V_m$, luego \[P_{R_i} \approx \dfrac{V_i^2}{R_i}\]
			
			
			Se demuestra que el campo de medida del aparato aumenta conforme aumenta la resistencia en serie con éste, aplicando el debido factor amplificador.
			
		\subsubsection{Divisores de tensión inductivos.}
			Sólo son válidos en corriente alterna, ya que están formados por dos bobinas acopladas en serie, en configuración, comunmente, de autotransformador. Comparados con los resistivos,
			apenas disipan energía, siempre
			que su R sea pequeña.
			
			
			Su impedancia aumenta con la
			frecuencia, por lo que no son útiles
			en circuitos de frecuencia variable. Se suelen utilizar a baja tensión y a frecuencia industrial.
			
			\begin{figure}[H]
				\centering
					\begin{circuitikz}
						\tikzstyle{every node}=[font=\large]
						\draw (5.25,19.5) to[L,l={ \large $R_SX_S$} ] (5.25,18);
						\draw (5.25,18) to[L,l={ \large $R_PX_P$} ] (5.25,16.5);
						\draw (5.25,19.75) to[short, -o] (4.5,19.75);
						\draw (5.25,16.25) to[short, -o] (4.5,16.25);
						\draw (5.25,19.75) to[short] (5.25,19.5);
						\draw (5.25,16.5) to[short] (5.25,16.25);
						\draw (5.25,16.25) to[short, -*] (7.5,16.25);
						\draw [-latex] (6.25,18) -- (5.25,18);
						\draw (6.25,18) to[short, -*] (7.5,18);
						\draw [-latex] (4.5,19.5) .. controls (4.5,17.75) and (4.5,18) .. (4.5,16.5) node[pos=0.4,left]{$U_1$};
						\draw [-latex] (7.5,17.75) -- (7.5,16.5)node[pos=0.5,right]{$U_2$};
						\draw [-] (9,18) -- (7.5,18);
						\draw [-] (9,16.25) -- (7.5,16.25);
						\draw (9,18) to[R,l={ \large $R_C$}] (9,16.25);
					\end{circuitikz}
			\end{figure}
		
		\subsubsection{Divisores de tensión capacitivos.}
			\begin{itemize}
				\item[-] Se utilizan tanto en corriente continua como en alterna, principalmente para la medida de altas tensiones.
				\item[-] En corriente continua los aparatos en paralelo con $C_P$ tienen que ser de tipo electrostático.
				\item[-] Su consumo de energía es nulo a efectos prácticos.
				\item[-] En c.a. se reduce su impedancia conforme aumenta la frecuencia, pudiendo llegar a comportarse como un cortocircuito.
				\item[-] Se pueden usar en B.T. para aplicaciones de bajo consumo.
			\end{itemize}
			
			\begin{figure}[H]
				\centering
					\begin{circuitikz}
						\tikzstyle{every node}=[font=\large]
						\draw (4.25,17.25) to[C,l={ \large $C_S$}] (4.25,15.75);
						\draw (4.25,15.75) to[C,l={ \large $C_P$}] (4.25,14.25);
						\draw  (7.5,15) circle (0.5cm) node {\large V} ;
						\draw (4.25,15.75) to[short] (7.5,15.75);
						\draw (7.5,15.75) to[short] (7.5,15.5);
						\draw (4.25,14.25) to[short] (7.5,14.25);
						\draw (7.5,14.25) to[short] (7.5,14.5);
						\draw (4.25,15.75) to[short, -*] (4.25,15.75);
						\draw (4.25,14.25) to[short, -*] (4.25,14.25);
						\draw (4.25,14.25) to[short, -o] (3,14.25);
						\draw (4.25,17.25) to[short, -o] (3,17.25);
						\draw [-latex] (3,17) -- (3,14.5)node[pos=0.5,left]{$U_1$};
						\draw (5.75,15.75) to[short, -*] (5.75,15.75);
						\draw (5.75,14.25) to[short, -*] (5.75,14.25);
						\draw [-latex] (5.75,15.5) -- (5.75,14.5)node[pos=0.5,right]{$U_2$};
					\end{circuitikz}
			\end{figure}
		
		\subsubsection{Divisores de intensidad resistivos.}
			Existen diversas alternativas a la hora de conectar un
			microamperímetro a resistencias en paralelo para ampliar su
			campo de medida y hacer que tenga varios alcances. No obstante, el montaje debe impedir siempre que por el elemento indicador circule la corriente principal (a medir) ya que ésta lo destruiría.
			
			\begin{figure}[H]
				\centering
					\begin{circuitikz}
						\tikzstyle{every node}=[font=\large]
						\draw (4.75,15.75) to[R,l={ \large $R_P$}] (6.75,15.75);
						\draw (4.75,17.5) to[R,l={ \large $R_m$}] (6.75,17.5);
						\draw  (5.75,17.5) circle (0.75cm);
						\draw [-latex] (4.25,17.5) -- (4.75,17.5)node[pos=0.6,above]{$I_m$};
						\draw (4.25,17.5) to[short] (4.25,15.75);
						\draw [-latex] (3.25,15.75) -- (3.75,15.75)node[pos=0.75,above]{I};
						\draw (3.75,15.75) to[short] (4.25,15.75);
						\draw (4.25,15.75) to[short, -*] (4.25,15.75);
						\draw (3.25,15.75) to[short, -o] (3,15.75);
						\draw [-latex] (4.25,15.75) -- (4.75,15.75)node[pos=0.9,above]{$I_d$};
						\draw (6.75,17.5) to[short] (7.25,17.5);
						\draw (7.25,17.5) to[short] (7.25,15.75);
						\draw (6.75,15.75) to[short] (8,15.75);
						\draw (8,15.75) to[short, -o] (8.25,15.75);
						\draw (7.25,15.75) to[short, -*] (7.25,15.75);
					\end{circuitikz}
			\end{figure}
			
		\subsubsection*{Shunt de Ayrton.}
			\begin{figure}[H]
				\centering
					\begin{circuitikz}
						\tikzstyle{every node}=[font=\large]
						\draw [line width=0.2pt, short] (8.75,15.5) -- (8.75,15)node[pos=1.75,below]{(-)};
						\draw [line width=0.2pt, short] (6.75,15.5) -- (6.75,15)node[pos=1.75,below]{$I_3$};
						\draw [line width=0.2pt, short] (4.75,15.5) -- (4.75,15)node[pos=1.75,below]{$I_2$};
						\draw [line width=0.2pt, short] (2.75,15.5) -- (2.75,15)node[pos=1.75,below]{$I_1$};
						\draw (4.75,17.5) to[R,l={ \large $R_m$}] (6.75,17.5);
						\draw  (5.75,17.5) circle (0.75cm);
						\draw [-latex] (4.25,17.5) -- (4.75,17.5)node[pos=0.6,above]{$I_m$};
						\draw (6.75,17.5) to[short] (7.25,17.5);
						\draw (2.75,15.75) to[R,l={ \large $R_1$}] (4.75,15.75);
						\draw (4.75,15.75) to[R,l={ \large $R_2$}] (6.75,15.75);
						\draw (6.75,15.75) to[R,l={ \large $R_3$}] (8.75,15.75);
						\draw (4.25,17.5) to[short] (2.75,17.5);
						\draw (2.75,17.5) to[short] (2.75,15.75);
						\draw (7.25,17.5) to[short] (8.75,17.5);
						\draw (8.75,17.5) to[short] (8.75,15.75);
						\draw (2.75,14.75) to[short, o-] (2.75,15.75);
						\draw (4.75,14.75) to[short, o-] (4.75,15.75);
						\draw (6.75,14.75) to[short, o-] (6.75,15.75);
						\draw (8.75,14.75) to[short, o-] (8.75,15.75);
						\draw (2.75,15.75) to[short, -*] (2.75,15.75);
						\draw (4.75,15.75) to[short, -*] (4.75,15.75);
						\draw (6.75,15.75) to[short, -*] (6.75,15.75);
						\draw (8.75,15.75) to[short, -*] (8.75,15.75);
					\end{circuitikz}
			\end{figure}
			
			Se inyecta corriente en alguno de los puntos $I_1, I_2$ ó $I_3$.
			\[I_3>I_2>I_1\]
			\[I_i=n_i\cdot I_m\ / i \in \mathbb{N}\]
			Para el campo de $I_1$: paralelo de $R_1, R_2$ y $R_3$ con $R_m$.
			\[(I_1-I_m)\cdot(R_1+R_2+R_3)=R_m\cdot I_m\Rightarrow R_P = R_1+R_2+R_3=\dfrac{R_m}{n_1-1}\]
			Para el campo de $I_2$: paralelo de $R_2$ y $R_3$ con $R_m + R_1$.
			Para el campo de $I_3$: paralelo de $R_3$ con $R_m + R_1 + R_2$.
			Resulta:
			\[R_2+R_3=\dfrac{R_P+R_m}{n_2}; R_3=\dfrac{R_P+R_m}{n_3}\]
			Para las potencias, si $I_i>\,>I_m\Rightarrow P_{R_i} \approx R_i\cdot I_i^2$.
			La $R$ interna del aparato es distinta para cada alcance.
			
		\subsubsection*{Otros acoplamientos de resistencias.}
			\begin{figure}[H]
				\centering
					\begin{circuitikz}
						\tikzstyle{every node}=[font=\large]
						\draw (4.75,19.5) to[R,l={ \large $R_m$}] (6.75,19.5);
						\draw  (5.75,19.5) circle (0.75cm);
						\draw [-latex] (4.25,19.5) -- (4.75,19.5)node[pos=0.6,above]{$I_m$};
						\draw (1.75,17) to[R,l={ \large $R_1$}] (3.75,17);
						\draw (3.75,17) to[R,l={ \large $R_2$}] (5.75,17);
						\draw (5.75,17) to[R,l={ \large $R_3$}] (7.75,17);
						\draw (1.75,17) to[short, -*] (1.75,17);
						\draw (3.75,17) to[short, -*] (3.75,17);
						\draw (5.75,17) to[short, -*] (5.75,17);
						\draw (7.75,17) to[short, -*] (7.75,17);
						\draw (4.25,19.5) to[short] (3.75,19.5);
						\draw (3.75,19) to[short, o-] (3.75,19.5);
						\draw (2.75,18.25) to[short, -o] (3.25,18.25);
						\draw (4.75,18.25) to[short, -o] (4.25,18.25);
						\draw (3.75,17.75) to[short, -o] (3.75,18);
						\draw [-latex] (3.75,19) -- (3.25,18.25);
						\draw [-latex, dashed] (3.75,19) -- (3.75,18);
						\draw [-latex, dashed] (3.75,19) -- (4.25,18.25);
						\node [font=\small] at (3.25,18.5) {1};
						\node [font=\small] at (4,17.75) {2};
						\node [font=\small] at (4.5,18.5) {3};
						\draw (3.75,17.75) to[short] (3.75,17);
						\draw (4.75,18.25) to[short] (5.75,18.25);
						\draw (5.75,18.25) to[short] (5.75,17);
						\draw (6.75,19.5) to[short] (7.75,19.5);
						\draw (7.75,19.5) to[short] (7.75,17);
						\draw (2.75,18.25) to[short] (1.75,18.25);
						\draw (1.75,18.25) to[short] (1.75,17);
						\draw [-latex] (0.5,17) -- (1.25,17)node[pos=0.85,above]{$I$};
						\draw (1.25,17) to[short] (1.75,17);
						\draw (0.5,17) to[short, -o] (0.25,17);
						\draw (7.75,17) to[short, -o] (8.5,17);
						\node [font=\large] at (9,17) {(-)};
						\node [font=\large] at (-0.25,17) {(+)};
					\end{circuitikz}
			\end{figure}
			
			\[I_3>I_2>I_1\]
			\[n_i=\dfrac{I_i}{I_m}\]
			
			Como $R_m>\,>R_i$, la $R$ interna global se mantiene prácticamente constante y con valor $R_1+R_2+R_3$.
			
	\subsection{Transformadores de medida.} 
		\subsubsection{Utilidad de los trafos de medida.}
			\begin{itemize}
				\item[-] Reducen a valores seguros (bajos) las magnitudes a medir. También pueden ampliarlas (poco normal).
				\item[-] Separan físicamente el equipo de medida del circuito a medir (aislamiento galvánico).
				\item[-] Permiten alejar el aparato de medida del punto donde se toma la señal a medir.
				\item[-] Los errores de inserción son más difíciles de evaluar.
			\end{itemize}
			
			\begin{figure}[H]
				\centering
				\resizebox{1\textwidth}{!}{%
					\begin{circuitikz}
						\tikzstyle{every node}=[font=\large]
						\draw [ ](3,21.25) to[L ] (3,19.25);
						\draw [ ](3.75,19.25) to[L ] (3.75,21.25);
						\draw [,  ](3.3,20.75) to[short] (3.3,19.75);
						\draw [,  ](3.45,20.75) to[short] (3.45,19.75);
						\draw [,   ] (5,20.25) circle (0.5cm) node {\large V} ;
						\draw [,  ](3.75,21.25) to[short] (5,21.25);
						\draw [,  ](5,21.25) to[short] (5,20.75);
						\draw [,  ](3.75,19.25) to[short] (5,19.25);
						\draw [,  ](5,19.75) to[short] (5,19.25);
						\draw [,  ](0.75,22.75) to[short] (0.75,18.5);
						\draw [,  ](2,22.75) to[short] (2,18.5);
						\draw[,  ] (3,21.25) to[short] (0.75,21.25);
						\draw[,  ] (3,19.25) to[short] (2,19.25);
						\draw (2,19.25) to[short, -*] (2,19.25);
						\draw (0.75,21.25) to[short, -*] (0.75,21.25);
						\node [font=\large] at (1.25,21.5) {A.T.};
						\node [font=\large] at (4.25,21.5) {B.T.};
						\draw [ ](5,19.25) to (5,19) node[ground]{};
						\draw (5,19.25) to[short, -*] (5,19.25);
						\ctikzset{resistor = european}
						\draw [,  ](7.5,19.25) to[R,l={ \large $Z$}] (9.75,19.25);
						\draw [,  ](7.5,19.25) to[short] (7.5,22.5);
						\draw [,  ](9.75,21.25) to[short] (9.75,22.5);
						\draw [ ](9.75,21.25) to[L ] (9.75,20.25);
						\draw [,  ](10.5,21.75) to[short] (11.5,21.75);
						\draw [,  ](10.5,19.75) to[short] (11.5,19.75);
						\draw [,   ] (12.5,21.75) circle (0.5cm) node {\large A} ;
						\draw [,  ](11.5,21.75) to[short] (12,21.75);
						\draw [,  ](13,21.75) to[short] (13.5,21.75);
						\draw [,  ](11.5,19.75) to[short] (13.5,19.75);
						\draw [,  ](13.5,21.75) to[short] (13.5,19.75);
						\draw [-latex] (9.75,22) -- (9.75,21.5)node[pos=0.5,left]{$I_1$};
						\draw [-latex] (11.25,21.75) -- (11.5,21.75)node[pos=0.5,above]{$I_2$};
						\draw [,  ](9.75,20.25) to[short] (9.75,19.25);
						\draw [ ](10.5,20.25) to[L ] (10.5,21.25);
						\draw [,  ](10.05,21.25) to[short] (10.05,20.25);
						\draw [,  ](10.2,21.25) to[short] (10.2,20.25);
						\draw [,  ](10.5,21.75) to[short] (10.5,21.25);
						\draw [,  ](10.5,20.25) to[short] (10.5,19.75);
					\end{circuitikz}
				}%
				\label{fig:my_label}
			\end{figure}
		
		\subsubsection{Transformadores de tensión.}
		
			Se basan en la relación que existe entre las tensiones del
			primario y el secundario en vacío:
			\[K_U=\dfrac{N_1}{N_2}\approx \dfrac{U_{1N}}{U_{2N}} \Rightarrow U_1=K_U\cdot U_2 \]
			
			
			Trabajan prácticamente en vacío. Sus cargas deben ser de
			muy bajo consumo (impedancias de voltímetros, bobinas
			voltimétricas de vatímetros, contadores...).
		
			
			Para mantener la linealidad hay que evitar en todo momento la saturación del núcleo (deben tenerse en cuenta los valores máximos de la onda senoidal).
			
			
			No exigen otros requerimientos especiales, pudiendo
			trabajar con el secundario abierto (sin carga).
			
			\subsubsection*{Tensiones nominales.}
				Las del primario ($U_{1N}$) suelen ir desde $2.2\,kV$ hasta $400\,kV$ , la del secundario ($U_{2N}$) suele ser $110\,V$, según la UNE-21127.
				
			\subsubsection*{Potencia nominal o de precisión.}
				Es la potencia aparente que, de forma permanente, pueden transferir al secundario, tanto a los aparatos como a los conductores de unión que tienen conectados, sin perder la \textbf{precisión} que le corresponde según su clase. Los valores preferentes suelen ser 10, 25, 50, 100, 200 y 500 $VA$.
				
			\subsubsection*{Impedancia de precisión.}
				Es una impedancia que colocada como carga en el
				secundario a la tensión nominal hace que el
				secundario, a la tensión nominal, hace que el
				transformador trabaje a su potencia nominal.
				
				
				Si se conecta al trafo una impedancia inferior a la de
				precisión se sobrepasa su potencia nominal de trabajo.
				
				Un transformador de tensión no debe trabajar nunca a
				potencia superior a la nominal ya que la relación $\dfrac{U_1}{U_2}$ se aleja de la relación de transformación nominal $K_U$, y las sobrecargas pueden producirle alteraciones y daños.
				
			\subsubsection*{Errores sistemáticos.}
				\begin{figure}[H]
					\centering
					\begin{circuitikz}
						\tikzstyle{every node}=[font=\large]
						\draw [-latex] (6.25,23.75) -- (5.25,25.25)node[pos=0.5,sloped,above]{$X_1·I_1$};
						\draw [-latex, dashed] (4.25,17.75) -- (5.25,20)node[pos=1,above]{$-I'_2$};
						\draw [-latex] (4.25,17.75) -- (7.75,17.75)node[pos=1,right]{$\Phi$};
						\draw [-latex] (4.25,17.75) -- (5.25,18.25)node[pos=1,right]{$I_0$};
						\draw [ color={rgb,255:red,0; green,0; blue,255}, -latex] (4.25,17.75) -- (6.25,20.5)node[pos=1,right]{$I_1$};
						\draw [dashed] (5.25,18.25) -- (6.25,20.5);
						\draw [dashed] (5.25,20) -- (6.25,20.5);
						\draw [ color={rgb,255:red,128; green,0; blue,255}, -latex] (4.25,17.75) -- (5.25,25.25)node[pos=1,left]{$U_1$};
						\draw [-latex, dashed] (4.25,17.75) -- (4.25,22.5)node[pos=1,left]{$E_1$};
						\draw [-latex] (4.25,22.5) -- (6.25,23.75)node[pos=0.7,sloped,above]{$R_1·I_1$};
						\draw [-latex, dashed] (4.25,17.75) -- (3,22.25)node[pos=1,left]{$-K_U·U_2$};
						\draw [-latex] (3.25,21.25) .. controls (3.75,21.75) and (4.25,21.75) .. (4.75,21.5)node[pos=0.3,above]{$\delta_U$};
						\draw [ color={rgb,255:red,255; green,0; blue,0}, -latex] (4.25,17.75) -- (5,15.5)node[pos=0.4,right]{$U_2$};
						\draw [ color={rgb,255:red,255; green,0; blue,0}, -latex] (5,15.5) -- (3.5,13.75)node[pos=0.5,sloped,above]{$R_2·I_2$};
						\draw [ color={rgb,255:red,255; green,0; blue,0}, -latex] (3.5,13.75) -- (4.25,13)node[pos=0.5,sloped,below]{$X_2·I_2$};
						\draw [-latex] (4.25,17.75) -- (4.25,13)node[pos=1,right]{$E_2$};
						\draw [ color={rgb,255:red,0; green,217; blue,0}, -latex] (4.25,17.75) -- (2,15.25)node[pos=1,left]{$I_2$};
						\draw [-latex] (3.25,16.5) .. controls (3.5,16) and (4.25,16) .. (4.75,16.25)node[pos=0.2,below]{$\varphi_2$};
					\end{circuitikz}
				\end{figure}
				
				Si el transformador fuese ideal la relación entre los fasores $\dfrac{U_1}{U_2}$ debería ser igual a $\dfrac{N_1}{N_2}$. Como esto no ocurre, se toma $K_U=\dfrac{N_1}{N_2} \approx \dfrac{U_1}{U_2}$.
				
				
				El alejamiento de la condición de idealidad se debe, principalmente, a las pérdidas en el hierro (magnéticas) y en el cobre (efecto Joule). Por ello, se consideran dos tipos de error: relación (\textbf{módulo}) y angular (\textbf{fase}).
				
				
				\subsubsection*{Error de relación o de tensión ($\varepsilon_U$).}
					Es la desviación de la tensión medida respecto al valor "real".
					\[K_U=\dfrac{U_{1N}}{U_{2N}}\]
					\[\Delta U = K_U\cdot U_2 - U_1\]
					\[\varepsilon_U\,[p.u.] = \dfrac{\Delta U}{U_1} = \dfrac{K_U\cdot U_2 - U_1}{U_1}\]
					
				\subsubsection*{Error angular o de fase ($\delta_U$).}
					Es la diferencia entre el ángulo de fase de $U_2$ y $U_1$ como consecuencia de la impedancia del transformador.
					\[\delta_U\,[rad] = \theta_2 - \theta_1\]
					
				
				La “clase de exactitud” depende de esos dos errores. Debido
				a su diseño para limitar los errores estos transformadores no
				son reversibles, ya que no mantienen la precisión.
				
				
				\subsubsection*{Error relativo complejo o compuesto ($\vec{\varepsilon}_U$).}
					\[\vec{\varepsilon}_U = \varepsilon_U + j\cdot \delta_U\]
					
					
					Como $\delta_U$ es muy pequeño se aproxima el ángulo por el
					valor de su seno:
					\[\delta_U \approx sen(\delta_U)\]
					
				
				\subsubsection*{Límites de error y aplicaciones según su clase.}
				\begin{table}[H]
					\begin{center}
						\begin{tabular}{clll}
							Clase & $\varepsilon_U\,[\%]$ & $\delta_U$ & Uso\\
							\hline
							0.1 & $\pm 0.1$ & $\pm 5'$ & Patrones\\
							0.2 & $\pm 0.2$ & $\pm 10'$ & Pruebas de precisión en lab.\\
							0.5 & $\pm 0.5$ & $\pm 30'$ & Pruebas ordinarias en lab.\\
							1   & $\pm 1.0$ & $\pm 1^\circ$ & Medidas rutinarias fuera de lab.\\
						\end{tabular}
					\end{center}
				\end{table}
					
			
		\subsubsection{Transformadores de intensidad.}
			Se basan en la relación que existe entre las intensidades
			del primario y del secundario:
			\[K_I=\dfrac{N_2}{N_1} \approx \dfrac{I_{1N}}{I_{2N}}\]
			
			
			El primario se pone en serie con el circuito, línea o carga
			cuya intensidad se quiere medir. Al secundario se conectan los circuitos amperimétricos
			de los equipos de medida, por lo que a efecto prácticos puede considerarse que trabajan en cortocircuito.
			
			
			Por cuestiones de precisión se les exige un buen
			comportamiento lineal. 
			
			
			Para evitar que los equipos resulten dañados por
			sobrecorrientes reflejadas en el secundario, se diseñan
			para que se saturen fácilmente.
			
			
			La saturación limita la máxima corriente secundaria a
			unas 5 veces la nominal (factor de seguridad).
			
			\subsubsection*{Intensidades nominales.}
				Primario: múltiplos de 5A. Secundario: está normalizada
				en 5A (excepcionalmente 1A).
				
			\subsubsection*{Potencia nominal o de precisión.}
				Potencia aparente máxima con la que se puede carga el
				secundario sin perder la precisión que indica su clase.
				
				
				Los valores más normales son: 2.5, 5, 10, 15 y 30 $VA$.
				
			\subsubsection*{Otros parámetros.}
				\begin{itemize}
					\item[-] \textit{Intensidad límite térmica:} máxima corriente primaria
					que pueden soportar durante un segundo, normalmente
					\[I_{th} = 100 \cdot I_n\]
					\item[-] \textit{Intensidad límite dinámica:} máxima corriente que
					pueden soportar sin sufrir deformación mecánica, normalmente
					\[I_{din}= 2.5 \cdot I_{th}\]
					\item[-] \textit{Factor de seguridad:} múltiplo ($n$) de la intensidad
					nominal del primario que provoca un error de relación
					del 10\%, normalmente
					\[n=5\]
				\end{itemize}
				
			\subsubsection*{Consideraciones sobre el diseño.}
				Para tener exactitud el flujo disperso debe ser muy bajo:
				\[\Phi_1 - \Phi_2 = \Phi_d \approx 0 \Rightarrow N_1 \cdot I_1 \approx N_2 \cdot I_2 \Rightarrow \dfrac{N_2}{N_1} \approx \dfrac{I_1}{I_2} = K_I\]
				
				
				Si el devanado secundario está abierto:
				\[I_2 = 0 \Rightarrow \Phi_1 = \Phi_d\]
				haciendo que toda la $I_1$ sea magnetizante, lo que hace crecer mucho el flujo que satura el núcleo y produce su calentamiento. Este aumento de flujo puede hacer que se induzcan tensiones muy elevadas en el secundario, por eso nunca debe quedar abierto el secundario en un transformador de intensidad.
				
			\subsubsection*{Errores sistemáticos.}
				Sus principales causas son el flujo de dispersión y la impedancia de carga del secundario. La relación $\dfrac{I_1}{I_2}$ debería mantenerse constante e igual a $\dfrac{N_2}{N_1}$. Como esta condición no se da, se toma
				\[K_I=\dfrac{N_2}{N_1} \approx \dfrac{I_1}{I_2}\]
				
				\begin{figure}[H]
					\centering
					\begin{circuitikz}
						\tikzstyle{every node}=[font=\large]
						\draw [, -latex, dashed] (4.25,17.75) -- (5.25,20)node[pos=1,sloped,above]{$-K_I \cdot I_2$};
						
						\draw [dashed] (4.25,17.75) -- (2,15.25);
						\draw [ color={rgb,255:red,0; green,217; blue,0}, -latex] (4.25,17.75) -- (3,15)node[pos=1,left]{$I_2$};
						
						\draw [, -latex] (4.25,17.75) -- (7.75,17.75)node[pos=1,right]{$\Phi_1$};
						\draw [, -latex] (4.25,17.75) -- (5.25,18.25)node[pos=1,right]{$I_0$};
						\draw [ color={rgb,255:red,0; green,0; blue,255}, , -latex] (4.25,17.75) -- (6.25,20.5)node[pos=1,right]{$I_1$};
						\draw [, dashed] (5.25,18.25) -- (6.25,20.5);
						\draw [, dashed] (5.25,20) -- (6.25,20.5);
						\draw [, -latex, dashed] (4.25,17.75) -- (4.25,21)node[pos=1,above]{$E_1$};
						\draw [, -latex] (4.25,17.75) -- (4.25,13)node[pos=1,right]{$U_2$};
						
						\draw [, -latex] (2,16.5) -- (2.5,16);
						\draw [, -latex] (4,15.25) -- (3.25,15.5);
						\draw [] (2.5,16) -- (3.25,15.5)node[pos=0.4,below]{$\delta_i$};
					\end{circuitikz}
				\end{figure}
				
				
				Se consideran los mismos tipos de error que para los trafos de tensión.
				
				
			\subsubsection*{Errores de relación ($\varepsilon_I$) y angular ($\delta_I$).}
				\[K_I=\dfrac{I_{1N}}{I_{2N}}\]
				\[\varepsilon_I\,[p.u.] = \dfrac{\Delta I}{I_1} = \dfrac{K_I\cdot I_2 - I_1}{I_1}\]	
				\[\delta_I\,[rad]= \theta_2 - \theta_1\]
				
			\subsubsection*{Error relativo complejo ($\vec{\varepsilon}_I$).}
				\[\vec{\varepsilon}_I = \varepsilon_I + j \cdot \delta_I\]
				
				
				Si la forma de onda es senoidal
				el error complejo coincide con
				el error compuesto.
			
				\begin{figure}[H]
					\centering
						\begin{circuitikz}
							\tikzstyle{every node}=[font=\large]
							\draw [, dashed] (3.25,6) -- (5.25,10.5)node[pos=0.5,sloped,below]{$\varepsilon_I$};
							\draw [, -latex, color = red] (1.25,5) -- (3.25,9.5)node[pos=1.05, sloped, above]{$K_I\cdot I_2$};
							\draw [-latex, color={rgb,255:red,0; green,0; blue,255}, , -] (1.25,5) -- (5.25,10.5)node[pos=1,right]{$I_1$};
							\draw [, -latex] (1.25,5) -- (3.25,6)node[pos=1,right]{$I_0$};
							\draw [, -latex] (1.25,5) -- (4.75,5);
							\draw [, -latex, dashed] (1.25,5) -- (1.25,10.25)node[pos=1,above]{$E_1$};
							\draw [, -latex, dashed] (3.25,9.5) -- (5.25,10.5)node[pos=0.5,above]{$\vec{\varepsilon_I}$};
							\draw [, short] (2.5,7.75) -- (3,7.5)node[pos=0.9,above]{$\delta_I$};
							\draw [, -latex] (2,8) -- (2.5,7.75);
							\draw [, -latex] (3.5,7.25) -- (3,7.5);
							\draw [, dashed] (3.25,9.5) -- (4.5,9);
							\draw [, short] (3.75,9.5) -- (4.75,10);
							\draw [, short] (4.75,10) -- (4.5,9.25);
							\draw [, short] (3.75,9.5) -- (4.5,9.25);
						\end{circuitikz}
				\end{figure}
				
			\subsubsection*{Límites del error según la clase.}
				Tienen las mismas clases que los trafos de tensión, pero cambian los límites de error admisibles. Los errores son en función del grado de carga.
			
				\begin{table}[H]
					\centering
					\begin{tabular}{|c|c|c|c|c||c|c|c|c|}
						\hline
						\multirow{2}{*}{Clase} & \multicolumn{4}{c||}{$\varepsilon_I$ (\%)} & \multicolumn{4}{c|}{$\delta_I$ (minutos)} \\
						\cline{2-9}
						& 5\% & 20\% & 100\% & 120\% & 5\% & 20\% & 100\% & 120\% \\
						\hline
						$0,1$ &$\pm0,4$ &$\pm0,2$ &$\pm0,1$ &$\pm0,1$ &$\pm15$ &$\pm8$ &$\pm5$ &$\pm5$ \\
						$0,2$ &$\pm0,75$& $\pm0,35$& $\pm0,2$& $\pm0,2$& $+30$ & $+15$ & $+10$ & $+10$ \\ 
						$0,5$ & $\pm1.5$ & $\pm.75$ & $\pm0.5$ & $\pm0.5$ & $\pm90$ & $\pm45$ & $\pm30$ & $\pm30$ \\ 
						$1.0$ & $\pm3.0$ & $\pm1.5$ & $\pm1.0$ & $\pm1.0$ & $\pm180$ & $\pm90$ & $\pm60$ & $\pm60$ \\
						\hline
					\end{tabular}
				\end{table}
				
			\subsubsection*{Vatímetro en conexión semidirecta.}
				La corriente I 2 es K I veces menor que la I 1, por lo que la
				potencia de Z es K I veces la indicada por el vatímetro.
			
				\begin{figure}[H]
					\centering
						\begin{circuitikz}
							\tikzstyle{every node}=[font=\large]
							\draw [, line width=0.5pt ] (1.75,4.75) circle (0.5cm) node {\large W} ;
							\draw [line width=0.5pt](1,3.25) to[L ] (2.5,3.25);
							\draw [line width=0.5pt](1,2.5) to[L ] (2.5,2.5);
							\draw [, line width=0.5pt](1.25,3) to[short] (2.25,3);
							\draw [, line width=0.5pt](1.25,2.8) to[short] (2.25,2.8);
							\draw [, line width=0.5pt](1.75,4.25) to[short] (1.75,3.75);
							\draw [, line width=0.5pt](1.75,3.75) to[short] (2.75,3.75);
							\draw [, line width=0.5pt](2.75,3.75) to[short] (2.75,3.25);
							\draw [, line width=0.5pt](2.75,3.25) to[short] (2.75,2.5);
							\draw [, line width=0.5pt](1.75,5.25) to[short] (1.75,5.75);
							\draw[, line width=0.5pt] (1.75,5.75) to[short] (0,5.75);
							\draw [, line width=0.5pt](0,5.75) to[short] (0,3.75);
							\draw [, line width=0.5pt](2.75,2.75) to[short] (2.75,1);
							\draw [, line width=0.5pt](-1,1) to[short] (3.5,1);
							\draw (2.75,1) to[short, -*] (2.75,1);
							\draw[, line width=0.5pt] (1.25,4.75) to[short] (0.5,4.75);
							\draw [, line width=0.5pt](0.5,4.75) to[short] (0.5,3.25);
							\draw [, line width=0.5pt](0.5,3.25) to[short] (1,3.25);
							\draw [, line width=0.5pt](2.25,4.75) to[short] (3,4.75);
							\draw [, line width=0.5pt](3,4.75) to[short] (3,3.25);
							\draw [, line width=0.5pt](2.75,3.25) to[short] (3,3.25);
							\draw [, line width=0.5pt](2.5,3.25) to[short] (2.75,3.25);
							\draw [, line width=0.5pt](-1,2.5) to[short] (1,2.5);
							\draw [, line width=0.5pt](2.5,2.5) to[short] (3.5,2.5);
							\draw [, line width=0.5pt](0,3.75) to[short] (0,2.5);
							\draw (0,2.5) to[short, -*] (0,2.5);
							\ctikzset{resistor = european}
							\draw [, line width=0.5pt](3.5,2.5) to[R,l={ \large $Z$}] (3.5,1);
							\draw [line width=0.5pt, -latex] (0.5,3.5) -- (0.5,4.25)node[pos=0.9,right]{$I_2$};
							\draw [line width=0.5pt, -latex] (-1,2.5) -- (-0.5,2.5)node[pos=0.9,above]{$I_1$};
							\node [font=\large] at (1.55,5.35) {*};
							\node [font=\large] at (1.1,4.9) {*};
						\end{circuitikz}
				\end{figure}
				
				\[P_Z=U_L\cdot I_1 \cdot \cos \varphi_1 \approx U_L\cdot K_I \cdot I_2 \cdot \cos \varphi_2 \approx K_I \cdot W\]
			
			\subsubsection*{Medida indirecta de corrientes.}
				Cuando no hay neutro una corriente puede obtenerse
				como “suma” de las otras dos, lo que no es posible
				midiendo de forma directa.
				\begin{figure}[H]
					\centering
						\begin{circuitikz}
							\tikzstyle{every node}=[font=\large]
							\draw  (15.5,15.25) circle (0.5cm) node {\large $I_A$} ;
							\draw (16,13.5) to[L] (15,13.5);
							\draw (15.07,13.5) to[short] (15.07,14.25);
							\draw (15.93,13.5) to[short] (15.93,14.25);
							\draw (15.07,14.25) to[short] (14.5,14.25);
							\draw (14.5,14.25) to[short] (14.5,15.25);
							\draw (14.5,15.25) to[short] (15,15.25);
							\draw (16,15.25) to[short] (16.5,15.25);
							\draw (16.5,15.25) to[short] (16.5,14.25);
							\draw (16.5,14.25) to[short] (15.93,14.25);
							\draw  (18,15.25) circle (0.5cm) node {\large $I_B$} ;
							\draw (18.5,12.75) to[L] (17.5,12.75);
							\draw (17.57,12.75) to[short] (17.57,14.25);
							\draw (18.43,12.75) to[short] (18.43,14.25);
							\draw (17.57,14.25) to[short] (17,14.25);
							\draw (17,14.25) to[short] (17,15.25);
							\draw (17,15.25) to[short] (17.5,15.25);
							\draw (18.5,15.25) to[short] (19,15.25);
							\draw (19,15.25) to[short] (19,14.25);
							\draw (19,14.25) to[short] (18.43,14.25);
							\draw  (20.5,15.25) circle (0.5cm) node {\large $I_C$} ;
							\draw (21,12) to[L] (20,12);
							\draw (20.07,12) to[short] (20.07,14.25);
							\draw (20.93,12) to[short] (20.93,14.25);
							\draw (20.07,14.25) to[short] (19.5,14.25);
							\draw (19.5,14.25) to[short] (19.5,15.25);
							\draw (19.5,15.25) to[short] (20,15.25);
							\draw (21,15.25) to[short] (21.5,15.25);
							\draw (21.5,15.25) to[short] (21.5,14.25);
							\draw (21.5,14.25) to[short] (20.93,14.25);
							\draw (14,13.5) to[short] (22,13.5);
							\draw (14,12.75) to[short] (22,12.75);
							\draw (14,12) to[short] (22,12);
							\node [font=\large] at (13.75,13.5) {A};
							\node [font=\large] at (13.75,12.75) {B};
							\node [font=\large] at (13.75,12) {C};
							\node [font=\large] at (13.75,11.25) {N};
							\draw (14,11.25) to[short] (22,11.25);
							\draw  (15.5,10) circle (0.5cm) node {\large $I_A$} ;
							\draw (16,8.25) to[L] (15,8.25);
							\draw (15.07,8.25) to[short] (15.07,9);
							\draw (15.07,9) to[short] (14.5,9);
							\draw (14.5,9) to[short] (14.5,10);
							\draw (14.5,10) to[short] (15,10);
							\draw (16,10) to[short] (16.5,10);
							\draw (16.5,10) to[short] (16.5,9);
							\draw  (18,10) circle (0.5cm) node {\large $I_B$} ;
							\draw (17,9) to[short] (17,10);
							\draw (17,10) to[short] (17.5,10);
							\draw (18.5,10) to[short] (19,10);
							\draw (19,10) to[short] (19,9);
							\draw  (20.5,10) circle (0.5cm) node {\large $I_C$} ;
							\draw (21,6.75) to[L] (20,6.75);
							\draw (19.5,9) to[short] (19.5,10);
							\draw (19.5,10) to[short] (20,10);
							\draw (21,10) to[short] (21.5,10);
							\draw (21.5,10) to[short] (21.5,9);
							\draw (14,8.25) to[short] (22,8.25);
							\draw (14,7.5) to[short] (22,7.5);
							\draw (14,6.75) to[short] (22,6.75);
							\node [font=\large] at (13.75,8.25) {A};
							\node [font=\large] at (13.75,7.5) {B};
							\node [font=\large] at (13.75,6.75) {C};
							\node [font=\large] at (13.75,6) {N};
							\draw (14,6) to[short] (22,6);
							\draw (15.93,8.25) to[short] (15.93,8.75);
							\draw (15.93,8.75) to[short] (20.93,8.75);
							\draw (20.93,8.75) to[short] (20.93,6.75);
							\draw (20.07,6.75) to[short] (20.07,9);
							\draw (20.07,9) to[short] (19.5,9);
							\draw (21.5,9) to[short] (21.5,8.5);
							\draw (21.5,8.5) to[short] (16.5,8.5);
							\draw (16.5,8.5) to[short] (16.5,9);
							\draw (19,9) to[short] (19,8.5);
							\draw (17,9) to[short] (17,8.75);
							\draw (17,8.75) to[short, -*] (17,8.75);
							\draw (19,8.5) to[short, -*] (19,8.5);
							\node [font=\large] at (14.85,13.6) {*};
							\node [font=\large] at (17.35,12.85) {*};
							\node [font=\large] at (19.85,12.1) {*};
							\node [font=\large] at (14.85,8.35) {*};
							\node [font=\large] at (19.85,6.85) {*};
						\end{circuitikz}
				\end{figure}
				
				En el esquema de la derecha:
				\[I_B = -(I_A + I_C)\]
				
				
				Un esquema similar puede desarrollarse
				para medir la $I_N$ en el circuito de la izquierda.
			
	\subsection{Sensores de efecto Hall.}
		\subsubsection{Aplicaciones y fundamento.}
			Si un conductor plano (sensor) por el que circula una
			corriente se somete a un campo magnético perpendicular
			a él las cargas se desvían debido a la fuerza que sobre
			ellas origina el campo.
			
			
			Pueden utilizarse indistintamente para adaptar tanto
			señales de tensión como de intensidad.
			\begin{figure}[H]
				\centering
					\begin{circuitikz}
						\tikzstyle{every node}=[font=\large]
						\draw [short] (8.75,12.25) -- (12.25,12.25);
						\draw [short] (10.75,10) -- (14.5,10);
						\draw [short] (8.75,12.25) -- (10.75,10);
						\draw [short] (12.25,12.25) -- (14.5,10);
						\draw [short] (8.75,12.25) -- (8.75,12);
						\draw [short] (8.75,12) -- (10.75,9.75);
						\draw [short] (10.75,10) -- (10.75,9.75);
						\draw [short] (10.75,9.75) -- (14.5,9.75);
						\draw [short] (14.5,10) -- (14.5,9.75);
						\draw [short] (13.5,8.75) -- (17.5,8.75);
						\draw [short] (17.5,8.75) -- (15.5,10.75);
						\draw [short] (13.5,8.75) -- (12.5,9.75);
						\draw [short] (10.5,12.25) -- (9.75,13);
						\draw [short] (9.75,13) -- (13.25,13);
						\draw [short] (13.25,13) -- (14.75,11.5);
						\draw (14.75,11.5) to[short, -o] (14.75,12.75);
						\draw (15.5,10.75) to[short, -o] (15.5,12.25);
						\draw (9.75,11) to[short, -o] (9,11);
						\draw (14.75,11) to (14.75,10.75) node[ground]{};
						\draw (13.5,11) to[short] (14.75,11);
						\draw [-latex] (9.5,11.75) -- (12.25,11.75);
						\draw [-latex] (9.75,11.5) -- (12.5,11.5);
						\draw [-latex] (10,11.25) -- (12.75,11.25);
						\draw [-latex] (10.25,11) -- (13,11);
						\draw [-latex] (10.5,10.75) -- (13.25,10.75);
						\draw [-latex] (10.75,10.5) -- (13.5,10.5);
						\draw [-latex] (11,10.25) -- (13.75,10.25);
						\draw [-latex] (9.25,12) -- (12,12);
						\node [font=\large] at (15.5,12.75) {$V=0$};
						\node [font=\large] at (9.5,12.5) {$\vec{E}$};
						\node [font=\large] at (8.75,11) {I};
					\end{circuitikz}
				\end{figure}
				
				\begin{figure}[H]
					\centering
					\begin{circuitikz}
						\draw [short] (8.75,12.25) -- (12.25,12.25);
						\draw [short] (10.75,10) -- (14.5,10);
						\draw [short] (8.75,12.25) -- (10.75,10);
						\draw [short] (12.25,12.25) -- (14.5,10);
						\draw [short] (8.75,12.25) -- (8.75,12);
						\draw [short] (8.75,12) -- (10.75,9.75);
						\draw [short] (10.75,10) -- (10.75,9.75);
						\draw [short] (10.75,9.75) -- (14.5,9.75);
						\draw [short] (14.5,10) -- (14.5,9.75);
						\draw [short] (13.5,8.75) -- (17.5,8.75);
						\draw [short] (17.5,8.75) -- (15.5,10.75);
						\draw [short] (13.5,8.75) -- (12.5,9.75);
						\draw [short] (10.5,12.25) -- (9.75,13);
						\draw [short] (9.75,13) -- (13.25,13);
						\draw [short] (13.25,13) -- (14.75,11.5);
						\draw (14.75,11.5) to[short, -o] (14.75,12.75);
						\draw (15.5,10.75) to[short, -o] (15.5,12.25);
						\draw (9.75,11) to[short, -o] (9,11);
						\draw (14.75,11) to (14.75,10.75) node[ground]{};
						\draw (13.5,11) to[short] (14.75,11);
						\draw [-latex] (9.5,11.75) .. controls (10.25,12) and (10.75,12) .. (12.25,11.75) ;
						\draw [-latex] (9.75,11.5) .. controls (10.75,11.75) and (11,11.75) .. (12.5,11.5) ;
						\draw [-latex] (10.25,11) .. controls (11.5,10.75) and (12,10.75) .. (13,11) ;
						\draw [-latex] (10.5,10.75) .. controls (11.75,10.5) and (12.25,10.5) .. (13.25,10.75) ;
						\draw [-latex] (10.75,10.5) .. controls (11.75,10.25) and (12.5,10.25) .. (13.5,10.5) ;
						\draw [-latex] (9.25,12) .. controls (10.5,12.25) and (10.5,12.25) .. (12,12) ;
						\node [font=\large] at (15.75,12.5) {$V=V_{Hall}$};
						\node [font=\large] at (9.5,12.5) {$(+)$};
						\draw [-latex] (10.75,13.75) -- (10.75,11.75);
						\draw [-latex] (11.25,13.25) -- (11.25,11.25);
						\draw [-latex] (11.75,12.75) -- (11.75,10.75);
						\draw [-latex] (10.75,9.75) -- (10.75,9.25);
						\draw [-latex] (11.25,9.75) -- (11.25,8.75);
						\draw [-latex] (11.75,9.75) -- (11.75,8.25);
						\node [font=\large] at (12.25,9.5) {$(-)$};
						\node [font=\large] at (10.5,14) {$\vec{B}$};
						\node [font=\large] at (8.75,11) {I};
					\end{circuitikz}
				\end{figure}
				
				La distorsión causada sobre la corriente origina una
				diferencia de potencial entre dos puntos perpendiculares
				al sentido de dicha corriente.
				
				
				Manteniendo constante la corriente por el sensor la
				tensión “Hall” es proporcional al campo:
				\[V_{Hall} = k \cdot I \cdot \beta \cdot \sen \theta\]
				
				
				Si el campo magnético es variable en el tiempo, la
				tensión Hall también lo será. Esto permite aprovechar este efecto para realizar medidas en corriente alterna y continua, o con cualquier forma de onda.
				
				
				La tensión Hall es muy débil. Como ejemplo, un campo
				de 1 Gauss ($10^{-4}\,T$) origina entre 20 y 30 $mV$. Sólo algunos materiales tienen interés para utilizarlos
				como sensores (placa conductora).
				
				
				Como todas las corrientes crean campos magnéticos, este
				efecto permite medirlas indirectamente. Debido a que es un valor débil, es necesario un circuito electrónico para adaptar y
				amplificar la tensión Hall.
				
				Haciendo que la corriente a través del sensor sea
				proporcional a la tensión de un circuito puede utilizarse
				este efecto para medir potencias.
				
		
		\subsubsection{Vatímetro de efecto Hall.}
			\begin{figure}[H]
				\centering
					\begin{circuitikz}
						\tikzstyle{every node}=[font=\large]
						\draw [short] (17.5,14) -- (19.75,14);
						\draw [short] (19.75,14) -- (21,15);
						\draw [short] (17.5,13) -- (19.75,13);
						\draw [short] (19.75,13) -- (21,14);
						\draw [short] (17.5,13) -- (18.75,14);
						\draw [short] (19.75,14) -- (21,14);
						\draw [short] (17.5,13) -- (17.5,12.75);
						\draw [short] (17.5,12.75) -- (19.75,12.75);
						\draw [short] (19.75,13) -- (19.75,12.75);
						\draw [short] (19.75,12.75) -- (21,13.75);
						\draw [short] (21,14) -- (21,13.75);
						\draw [short] (17.5,12) -- (19.75,12);
						\draw [short] (19.75,12) -- (21,13);
						\draw [short] (21,13) -- (20,13);
						\draw [short] (17.5,12) -- (18.5,12.75);
						\draw [short] (19.75,14) -- (19.75,16);
						\draw [short] (21,15) -- (21,17);
						\draw [short] (17.5,14) -- (17.5,16);
						\draw [short] (17.5,16) -- (15,16);
						\draw [short] (19.75,16) -- (19.75,18);
						\draw [short] (19.75,18) -- (15,18);
						\draw [short] (21,17) -- (21,19);
						\draw [short] (19.75,18) -- (21,19);
						\draw [short] (21,19) -- (16,19);
						\draw [short] (15,16) -- (15,10.25);
						\draw [short] (17.5,12) -- (17.5,10.5);
						\draw [short] (19.75,12) -- (19.75,10.25);
						\draw [short] (17.5,10.5) -- (17.5,10.25);
						\draw [short] (21,13) -- (21,11.25);
						\draw [short] (17.5,10.25) -- (15,10.25);
						\draw [short] (19.75,10.25) -- (19.75,8.75);
						\draw [short] (15,18) -- (13,18);
						\draw [short] (13,18) -- (13,8.5);
						\draw [short] (13,8.5) -- (19.75,8.5);
						\draw [short] (19.75,8.5) -- (19.75,8.75);
						\draw [short] (21,11.25) -- (21,9.5);
						\draw [short] (19.75,8.5) -- (21,9.5);
						\draw [short] (13,18) -- (14.5,19);
						\draw [short] (16,19) -- (14.5,19);
						\draw [short] (15,10.25) -- (16.25,11.25);
						\draw [short] (16.25,11.25) -- (16.25,16);
						\draw [short] (16.25,11.25) -- (17.5,11.25);
						\draw [short] (18.25,13.25) -- (18.25,14);
						\draw [short] (19.5,13.25) -- (19.5,14);
						\draw [short] (20.25,13.75) -- (20.25,14.4);
						\draw [short] (19,13.75) -- (19,14);
						\draw [-latex] (18.25,12.75) -- (18.25,12.25);
						\draw [-latex] (19.5,12.75) -- (19.5,12.25);
						\draw [-latex] (20.25,13.15) -- (20.25,12.75);
						\node [font=\large] at (18.75,13.5) {$\Phi_m$};
						\draw (20.25,13.25) to[short] (21.5,13.25);
						\draw (17.75,13.25) to[short] (17.25,13.25);
						\draw (17.75,13.25) to[short] (17.25,13.25);
						\draw (17.75,13.25) to[short] (17.25,13.25);
						\draw (17.75,13.25) to[short] (17.25,13.25);
						\draw (10.25,10.25) to[R,l={ \large $R_C$}] (12.25,10.25);
						\draw (10.25,8.75) to[L ] (10.25,10.25);
						\draw (9.5,10.25) to[L ] (9.5,8.75);
						\draw [short] (9.8,10) -- (9.8,9);
						\draw [short] (9.95,10) -- (9.95,9);
						\draw [short] (11,14.75) -- (15,14.75);
						\draw [short] (15,14.25) -- (16.25,15);
						\draw [short] (13,13.75) -- (15,13.75);
						\draw [short] (15,13.75) -- (16.25,14.5);
						\draw [short] (13,13.25) -- (15,13.25);
						\draw [short] (15,13.25) -- (16.25,14);
						\draw [short] (13,12.75) -- (15,12.75);
						\draw [short] (15,12.75) -- (16.25,13.5);
						\draw [short] (13,12.25) -- (15,12.25);
						\draw [short] (15,12.25) -- (16.25,13);
						\draw [short] (13,11.75) -- (11.25,11.75);
						\draw [short] (15,14.25) -- (13,14.25);
						\draw [short] (15,14.75) -- (16.25,15.5);
						\draw (8.25,8.75) to[short] (8.25,14.75);
						\draw (8.25,7.75) to[short, o-] (8.25,8.25);
						\draw (9,7.75) to[short, o-] (9,8.25);
						\draw (11.25,11.75) to[short] (10.25,11.75);
						\draw [short] (17.25,13.25) -- (12.5,10.25);
						\draw (12.25,10.25) to[short] (12.5,10.25);
						\draw [short] (21.5,13.25) -- (16,8.75);
						\draw [short] (16,8.75) -- (10.25,8.75);
						\draw  (23.25,13.5) circle (0.5cm) node {\large W} ;
						\draw [short] (20,14) -- (21,14.75);
						\draw [short] (21,14.75) -- (23.25,14.75);
						\draw [short] (23.25,14.75) -- (23.25,14);
						\draw [short] (18.55,12.85) -- (17.75,12.25);
						\draw [short] (17.75,12.25) -- (23.25,12.25);
						\draw [short] (23.25,13) -- (23.25,12.25);
						\draw [-latex] (11.25,14.75) -- (11.75,14.75)node[pos=0.8,above]{$I_L$};
						\draw (8.75,14.75) to[R,l={ \large $R_L$}] (10.25,14.75);
						\draw (10.25,11.75) to[short] (9,11.75);
						\draw (9,11.75) to[short] (9,8.25);
						\draw (9.5,10.25) to[short] (9,10.25);
						\draw (9.5,8.75) to[short] (8.25,8.75);
						\draw (8.25,8.25) to[short] (8.25,8.75);
						\draw (8.25,8.75) to[short, -*] (8.25,8.75);
						\draw (9,10.25) to[short, -*] (9,10.25);
						\draw (8.25,14.75) to[short] (8.75,14.75);
						\draw (10.25,14.75) to[short] (11,14.75);
						\node [font=\large] at (8.6,7.25) {Línea};
						\draw [-latex] (13.5,14.25) -- (14,14.25);
						\draw [-latex] (13.5,13.75) -- (14,13.75);
						\draw [-latex] (13.5,13.25) -- (14,13.25);
						\draw [-latex] (13.5,12.75) -- (14,12.75);
						\draw [-latex] (13.5,12.25) -- (14,12.25);
						\draw [-latex] (11.75,11.75) -- (11.25,11.75);
					\end{circuitikz}
			\end{figure}
			
			\[P_W = I_L \cdot V_L \cdot \cos{\phi}\]